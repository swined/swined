\documentclass[russian]{article}
\usepackage[T1]{fontenc}
\usepackage[utf8]{inputenc}
\usepackage{amsmath, amssymb}
\usepackage{esint}
\usepackage{babel}
\makeatletter
\makeatother

\begin{document}

\section*{3.1}

$x(t)=\frac{1}{2}\int_0^t y(s)(t-s)^2 d s + 1$
\footnote{
	$x(t) = \frac{1}{(n-1)!} \int_a^t y(s)(t-s)^{n-1}ds$
}

$x'(t) = \int_0^t y(s)(t-s)ds$
\footnote{
	$\frac{d}{d x} \int_{\beta(x)}^{\alpha(x)}f(x,y) d y = \int_{\beta(x)}^{\alpha(x)}\frac{\partial}{\partial x}f(x, y) d y + \alpha ' (x) f(x,\alpha(x)) - \beta ' (x) f(x, \beta (x))$
}

$x''(t) = \int_0^t y(s) ds$

$x'''(t) = y(s)$

$y(t) + t[\frac{1}{2}\int_0^t y(s)(t-s)^2 ds + 1] = e^t$

\section*{3.2}

\paragraph*{а}

$\varphi(t) = \int _ 0 ^ \pi \sin(t-2s)\varphi(s)ds + \cos 2t$

$\varphi(t) = \sin t \int _0^\pi \cos 2s \varphi(s) ds - \cos t \int _0^\pi \sin 2s \varphi(s) ds + \cos 2t$

$q_1 = \int _0^\pi \cos 2s \varphi(s) ds$
$q_2 = \int _0^\pi \sin 2s \varphi(s) ds$

$\varphi(t) = q_1 \sin t - q_2 \cos t + \cos 2t$

$\varphi(t) = \sin t \int _0^\pi \cos 2s (q_1 \sin s - q_2 \cos s + \cos 2s) ds - \cos t \int _0^\pi \sin 2s (q_1 \sin s - q_2 \cos s + \cos 2s) ds + \cos 2t$

$\begin{cases}
q_1 = q_1 \int _0^\pi \cos 2s \sin s ds - q_2 \int_0^\pi \cos 2s \cos s ds + \int _0^\pi       \cos ^2 2s ds\\
q_2 = q_1 \int _0^\pi \sin 2s \sin s ds - q_2 \int _0^\pi \sin 2s \cos s ds + \int _0^\pi \sin 2s \cos 2s ds
\end{cases}$

$\int _0^\pi \cos 2s \sin s ds = - \frac{2}{3}$

$\int_0^\pi \cos 2s \cos s ds = 0 $

$\int _0^\pi \cos ^2 2s ds = \frac{\pi}{2}$

$\int _0^\pi \sin 2s \sin s ds = 0 $

$\int _0^\pi \sin 2s \cos s ds = \frac{4}{3} $

$\int _0^\pi \sin 2s \cos 2s ds = 0$

$\begin{cases}
q_1 = - \frac{2}{3} q_1 + \frac{\pi}{2}\\
q_2 =  - \frac{4}{3} q_2 
\end{cases}$

$\begin{cases}
q_1 = \frac{3\pi}{10}\\
q_2 = 0
\end{cases}$

$\varphi(t) = \frac{3\pi}{10} \sin t + \cos 2t$

\paragraph*{б}

$\varphi(t) = \int_{-1}^1 [t s + t^2 s^2] \varphi(s) ds + t^2 - t^4$

$\varphi(t) = t \int_{-1}^1 s \varphi(s) ds + t^2 \int_{-1}^1 s^2 \varphi(s) ds + t^2 - t^4$

$q_1 = \int_{-1}^1 s \varphi(s) ds$

$q_2 = \int_{-1}^1 s^2 \varphi(s) ds$

$\varphi(t) = t q_1 + t^2 q_2 + t^2 - t^4$

$\varphi(t) = t \int_{-1}^1 s (s q_1 + s^2 q_2 + s^2 - s^4) ds + t^2 \int_{-1}^1 s^2 (s q_1 + s^2 q_2 + s^2 - s^4) ds + t^2 - t^4$

$t q_1 + t^2 q_2 + t^2 - t^4 = t \int_{-1}^1 s (s q_1 + s^2 q_2 + s^2 - s^4) ds + t^2 \int_{-1}^1 s^2 (s q_1 + s^2 q_2 + s^2 - s^4) ds + t^2 - t^4$

$\begin{cases}
q_1 = \int_{-1}^1 s (s q_1 + s^2 q_2 + s^2 - s^4) ds \\
q_2 = \int_{-1}^1 s^2 (s q_1 + s^2 q_2 + s^2 - s^4) ds
\end{cases}$

$\begin{cases}
q_1 = 4 q_1 - 8 \\
q_2 = 8 (q_2+1) - 12
\end{cases}$

$\begin{cases}
q_1 = \frac{8}{3} \\
q_2 = \frac{4}{7}
\end{cases}$

$\varphi(t) = t \frac{8}{3} + t^2 \frac{4}{7} + t^2 - t^4$

\section*{3.3}

$x(t) - \frac{\ln 2}{2}\int_0^1 2^{t+s} x(s) ds = t$

$(I-\mu A)x = f$

$||A||=\sup_{x \ne 0} \frac{||Ax||}{||x||} = \sup_{x \ne 0} \frac{\sqrt{\int_0^1 |Ax(t)|^2 dt}}{||x||}=\sup_{x \ne 0} \frac{\sqrt{\int_0^1 |\int_0^1 2^{t+s}x(s)ds|^2 dt}}{||x||}$ 

по неравенству Коши-Буняковского \footnote{
	$\int_0^1|f(t) g(t)| dt \le ||f|| ||g|| $
}
$||A|| \le \sup_{x \ne 0} \frac{\sqrt{\int_0^1 ||2^{t+s}||^2 ||x||^2 dt}}{||x||} = \sqrt{\int_0^1 ||2^{t+s}||^2 dt} = \sqrt{\int_0^1 [\int_0^1 2^{t+s} ds]dt} =\sqrt{\frac{1}{\ln 2} \int_0^1 2^t dt} = \frac{1}{\ln 2}$

т.к. $||\mu A|| \le \frac{1}{2} < 1$, то по теореме Неймана \footnote{
	если $||B|| < 1$, то оператор $I-B$ обратим и $(I-B)^{-1} = \sum _{n=0}^\infty B^n$, где $B^0 = I$, $B^{n+1} = B B^n$
}
$x=(I-\mu A)^{-1}f=\sum_{n=0}^{\infty}\mu^n A^n f$

по теореме о повторном ядре \footnote{
	если $(Ax)(t)=\int_a^bK(t,s)x(s)ds$ (оператор Гильберта-Шмидта)
	и $\iint_{a}^{b}|K(t,s)|^2 dt ds < \infty$, то $A^n$ - оператор Гильберта-Шмидта 
	и $(A^n x)(t) = \int_a^b K_n(t,s)x(s) ds$, где $K_n(t,s) = \int_a^b K(t,r) K_{n-1}(r,s)dr$ - повторное ядро
} $x=t + \sum_{n=1}^{\infty}\mu^n \int_0^1 K_n(t,s)s ds=t+\int_0^1[\sum_{n=1}^\infty \mu ^n K_n(t,s)]s ds$

$K_1 = 2^{t+s}$

$K_2 = \int_0^1 2^{t+r} 2^{r+s} dr = 2^{t+s} \int_0^1 2^{2r} dr = \frac{3}{2 \ln 2} 2^{t+s}$

$K_n = (\frac{3}{2 \ln 2})^{n-1} 2^{t+s} $

$R(t,s) = \sum_{n=1}^\infty (\frac{\ln 2}{2})^n (\frac{3}{2 \ln 2})^{n-1} 2^{t+s} = \frac{\ln 2}{2} 2^{t+s} \sum_{n=0}^\infty (\frac{3}{4})^n = 2 \ln 2 2^{t+s}$

$x=t+\int_0^1 s R(t,s) ds$

\section*{3.4}

$A \to K(t,s)=4 t s^2$

по теореме об операторе сопряженном оператору ГШ \footnote{
	если $A$ - оператор ГШ с ядром $K(t,s)$, то $A*$ - оператор ГШ и $K*(t,s)=\overline{K(s,t}$
}
$A* \to K(t,s)=4 s t^2$

$y(t) - 4 \int_0^1 s t^2 y(s) ds = 0$

$y(t) = t^2$

по теореме Фредгольма \footnote{
	пусть $H$ - гильбертово пространство, $A:H \to H$ - линейный компактный оператор. разрешимость $x-Ax=f$ устанавливается с помощью однородного уравнения $x-Ax=0$ и однородного сопряженного уравнения $y-A*y=0$:
	
	для уравнения (н) возможны два случая:
	
	1. если (со) имеет только нулевое решение, то (н) имеет единственное решение для любого $f$
	
	2. если (со) имеет $n$ линейно-независимых решений $\{x_n\}$ и (о) имеет ровно $n$ линейно-незавимых решений $\{y_n\}$, то (н) разрешимо если и только если $(y_k,f) = 0 \forall k\in[1,n]$
}
уравнение разрешимо если и только если $\int_0^1f(t)t^2 = 0$

\section*{3.5}

достаточное условие экстремальности \footnote{
	если $y$ - экстремаль функционала $L(y)=\int_a^b F(x,y(x),y'(x))dx$ при $y(a)=y_0$, $y(b)=y_1$, то $y$ - решение уравнения Эйлер $\frac{\partial F}{\partial y} - \frac{d}{dx}\frac{\partial F}{\partial y'} = 0$
}

\paragraph*{а}

$F_y = 2y+2e^x$

$F_{y'} = 2y'(x)$

$\frac{d}{dx}F_{y'} = 2y''(x)$

$y'' - y = e^x$

$y=c e^{\lambda x}$

$(\lambda^2-1)e^{\lambda x} = 0 \Rightarrow \lambda = \pm 1$

$y=C_1 e^x + C_2 e^{-x}$

частное решение неоднородного:

$y=C_1(x) e^x + C_2(x) e^{-x} \Rightarrow y= \frac{1}{2}x e^x $

$y' = C_1' e^x +C_1 e^x + C_2' e^{-x} - C_2 e^{-x}$

$y'' = C_1'' e^x + 2 C_1' e^x + C_1 e^x + C_2'' e^{-x} - 2 C_2 ' e^{-x} + C_2 e^{-x}$

$(C_1'' + 2C_1' + C_1 - C_1) e^x - (C_2''-2C_2' + C_2 - C_2) e^{-x} = e^x$

$\begin{cases}
C_1 = \frac{1}{2}x \\
C_1' = \frac{1}{2} \\
C_1'' = 0 \\
C_2 = 0
\end{cases}$

общее решение неоднородного:

$y=C_1 e^x + C_2 e^{-x} + \frac{1}{2}x e^x$

$\begin{cases}
y(0) = C_1 + C_2 = \frac{1}{2} \\
y(1) = C_1 e + C_2 \frac{1}{e} + \frac{1}{2} e = e
\end{cases}
\Rightarrow 
\begin{cases}
C_1 = \frac{1}{2} \\
C_2 = 0
\end{cases}$

$y(x)=\frac{1}{2}(x+1)e^x$

\paragraph*{б}

$F = 2xy + (x^2 + e^y)y'$

$F_y = 2x + e^y y'$

$F_{y'} = x^2 + e^y$

$\frac{d}{dx}F_{y'} = 2x + y'e^y$

$2 e^y y' = 0$

$y' = 0$

$y= c_1 x + c_2 $

$\begin{cases}
y(x_0) = c_1 x_0 + c_2 = y_0 \\
y(x_1) = c_1 x_1 + c_2 = y_1
\end{cases}
\Rightarrow
\begin{cases}
c_1 = \frac{x_0 - x_1}{y_0 - y_1}\\
c_2 = y_0 - \frac{x_0 - x_1}{y_0 - y_1}  x_0
\end{cases}$

\paragraph*{в}

$F = (y'')^2 - y^2 + x^2 $

$F_y = -2y$

$F_{y'} = 0$

$F_{y''} = 2 y''$

$y'''' - y= 0$

$y=e^{\lambda x} \Rightarrow \lambda ^4 e^{\lambda x} - e^{
lambda x} = 0 \Rightarrow \lambda^4 = 1 \Rightarrow 
lambda = \pm 1, \pm i$

$y=c_1 e^x + c_2 e^{-x} + c_3 e^{i x} + c_4 e^{-i x}$

$y=c_1 e^x - c_2 e^{-x} + i c_3 e^{i x} - i c_4 e^{-i x}$

$\begin{cases}
y(0) = c_1 + c_2 + c_3 + c_4 = 1\\
y'(0) = c_1 - c_2 + i c_3 - i c_4 = 0\\
y(\frac{\pi}{2}) = c_1 e^\frac{\pi}{2} + c_2 e^{-\frac{\pi}{2}} + i c_3 - i c_4 = 0\\
y'(\frac{\pi}{2}) = c_1 e^\frac{\pi}{2} - c_2 e^{-\frac{\pi}{2}} - c_3 - c_4 = -1\\
\end{cases}
\Rightarrow
\begin{cases}
c_1 = c_2 = 0
c_3 = c_4 = \frac{1}{2}
\end{cases}$

$y= \cos x$

\paragraph*{г}

$ F = 2z - 4y^2 + y'^2 - z'^2 $

$\begin{cases}
4 y + y'' = 0 \\
1 + z'' = 0
\end{cases}$

$\begin{cases}
y = \sin 2x \\
z = -\frac{1}{2} x^2 + \frac{\pi}{4} x + \frac{4}{\pi}
\end{cases}$

\section*{3.6}

условие трансверсальности \footnote{
	$F(x_1)-F_{y'}(x_1)[y'(x_1)-\varphi'(x_1)] = 0$
}
для функционалов вида $F=\int f(x,y)\sqrt{1+y'^2}$, где $f(x,y) \ne 0$:

$f(x_1, y(x_1))\sqrt{1+[y'(x_1)]^2} - f(x_1, y(x_1)) \frac{2y'(x_1)}{2\sqrt{1 + [y'(x_1)]^2}}[y'(x_1) - \varphi'(x_1)] = 0$

$y'(x_1)\varphi'(x_1)=-1$

$T=\int_{x_0}^{x^1}\frac{\sqrt{1+(y'(x))^2}}{\sqrt{2g(y_0-y(x))}}$

$\begin{cases}
x=\frac{c_1}{2}(t-\sin t) + C_2 \\
y = y_0 - \frac{c_1}{2}(1 - \cos t)
\end{cases}$

$\begin{cases}
0 = C_2 \\
0 = y_0
\end{cases}$

$\frac{dy}{dx} = \frac{dy}{dt}\frac{dt}{dx} = \frac{\sin t}{\cos t - 1}$

\paragraph*{а}

$-2 \frac{\sin t}{\cos t - 1} = -1$

$2 \sin t = \cos t - 1$

$\frac{2}{\sqrt{5}} \sin t - \frac{1}{\sqrt{5}}\cos t = - \frac{1}{\sqrt{5}}$

$\sin (\alpha + t_1)=-\frac{1}{\sqrt{5}} \Rightarrow t_1 = -2 arcsin \frac{1}{\sqrt{5}}
\Rightarrow 
\begin{cases}
\sin t_1 = \frac{4}{5} \\
\cos t_1 = \frac{3}{5} \\
\end{cases}$

$ 2x(t_1)+y(t_1) = 1 \Rightarrow 	c_1 = -\frac{5}{20 arcsin \frac{1}{\sqrt{5}} + 9}$

\paragraph*{б}

$\begin{cases}
(x(t_1) - 9)^2 + y^2(t_1)=9 \\
-y(t_1) x'(t_1) + (9 - x(t_1))y'(t_1) = 0
\end{cases}$

$\begin{cases}
(\frac{c_1}{2}(t_1-\sin t_1) - 9)^2 + [\frac{c_1}{2}(\cos t_1 - 1)]^2=9 \\
-\frac{c_1}{2}(\cos t_1 - 1) \frac{c_1}{2}(1-\cos t_1) - (9 - \frac{c_1}{2}(t_1-\sin t_1))\frac{c_1}{2}\sin t_1 = 0
\end{cases}$

\end{document}

