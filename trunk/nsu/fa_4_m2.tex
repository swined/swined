\documentclass[russian]{article}
\usepackage[T1]{fontenc}
\usepackage[utf8]{inputenc}
\usepackage{amssymb}
\usepackage{esint}
\usepackage{babel}
\makeatletter
\makeatother

\begin{document}
2.1

а) $A(a\varphi_1+b\varphi_2)=(a\varphi_1+b\varphi_2)(x^\alpha)=a\varphi_1(x^\alpha)+b\varphi_2(x^\alpha)=aA\varphi_1+bA\varphi_2$

б) $A$ непрерывен как суперпозиция двух непрерывных функций

$||\varphi||_{C[0,1]}=\sup_{x\in[0,1]}|\varphi(x)|$

$||A\varphi||_{C[0,1]}=\sup_{x\in[0,1]}|\varphi(x^\alpha)|=_{t=x^\alpha}\sup_{t\in[0,1]}|\varphi(t)|=||\varphi||_{C[0,1]}$

в) $||A\varphi||^2=\int_0^1|\varphi(x^\alpha)|^2dx=_{t=x^\alpha}\frac{1}{\alpha}\int_0^1|\varphi(t)|^2t^{\frac{1}{\alpha}-1}dt$

$\frac{1}{\alpha}-1\ge0\Rightarrow\int_0^1|\varphi(t)|^2t^{\frac{1}{\alpha}-1}dt\le\int_0^1|\varphi(t)|^2dt<+\infty$

$\alpha>1\Rightarrow\varphi(x)=x^\beta\in L_2[0,1]\Leftrightarrow\int_0^1|x^\beta|^2dx=\int_0^1x^{2\beta}dx=\frac{x^{2\beta+1}}{2\beta+1}|_0^1=\frac{1}{2\beta+1}<_{2\beta+1>0}+\infty$

$A\varphi\in L_2[0,1] \Leftrightarrow \int_0^1|(x^\alpha)^\beta|^2dx=\int_0^1x^{2\alpha\beta}dx=\frac{x^{2\alpha\beta+1}}{2\alpha\beta+1}|_0^1<+\infty \Leftrightarrow 2\alpha\beta+1>0 \Rightarrow \beta > -\frac{1}{2\alpha}$

г) $ ||A||=\sup_{x\ne0}\frac{||Ax||}{||x||}=\sup_{x\ne0}\frac{\sqrt{\int_0^1|Ax(t)|^2dt}}{\sqrt{\int_0^1|x(t)|^2dt}}=\sup_{x\ne0}\frac{\sqrt{\int_0^1|x(t^\alpha)|^2dt}}{\sqrt{\int_0^1|x(t)|^2dt}}=\sup_{x\ne0}\frac{\sqrt{\int_0^1|x(u)|^2\frac{1}{\alpha}u^{\frac{1}{\alpha}-1}du}}{\sqrt{\int_0^1|x(t)|^2dt}} $

$ 0 < \alpha \le 1 \Rightarrow u^{\frac{1}{\alpha}-1} \le 1 \forall u \in [0,1] \Rightarrow ||A|| \le \frac{1}{\sqrt{\alpha}} \sup_{x\ne0}\frac{\sqrt{\int_0^1|x(u)|^2du}}{\sqrt{\int_0^1|x(t)|^2dt}}=\frac{1}{\sqrt{\alpha}}$

$ x(t)=t^\beta : ||A|| \ge \sup_{\beta}\frac{\sqrt{\int_0^1u^{2\beta}\frac{1}{\alpha}u^{\frac{1}{\alpha}-1}}}{\sqrt{\int_0^1t^{2\beta}dt}} = \frac{1}{\sqrt{\alpha}}\sup_{\beta}\frac{\sqrt{\int_0^1u^{2\beta+\frac{1}{\alpha}-1}}}{\sqrt{\int_0^1t^{2\beta}dt}}$

$ 2 \beta + 1 > 0 : \int_0^1t^{2\beta}dt = \frac{1}{2\beta+1} $

$ 2\beta + \frac{1}{\alpha} > 0 : \int_0^1u^{2\beta+\frac{1}{\alpha}-1} = \frac{1}{2\beta+\frac{1}{\alpha}} $

$ ||A|| \ge \frac{1}{\alpha} \sup _ \beta \frac{\sqrt{2\beta+1}}{\sqrt{2\beta+\frac{1}{\alpha}}} \ge_{\beta \rightarrow \infty } \frac{1}{\sqrt{\alpha}}$

$ ||A|| = \frac{1}{\sqrt{\alpha}} $

2.2

а) $ ||A\varphi||^2=\int_0^1|A\varphi(q)|^2dq=\int_{0}^{1}|q\varphi(q)|^2dq \le \int{}_{0}^{1}|\varphi (q)|^2dq =|| \varphi ||^2<+\infty$ т.к. $\varphi \in L_2 $

$ ||A|| = \sup _{\varphi \ne 0 } \frac{||A\varphi||}{||\varphi||} \le 1 $ 

$ ||A|| = \sup _{\varphi \ne 0 } \frac{||A\varphi||}{||\varphi||} \ge _{\varphi=q^\beta} \sup_\beta \frac{\sqrt{\int_0^1q^2q^{2\beta}dq}}{\sqrt{\int_0^1q^{2\beta}dq}} = \sup _\beta \frac{\sqrt{2\beta + 1}}{\sqrt{2\beta+3}} \ge _{\beta \rightarrow \infty} 1$

$ ||A|| = 1 $

б) $q\varphi(q)-\lambda\varphi(q)=0\Rightarrow(q-\lambda)\varphi(q)=0\Rightarrow\varphi=_{L_{2}}0 \Rightarrow $ ненулевых решений нет $\Rightarrow$ $\sigma_{p}(A)=\varnothing$

в) $ \lambda \in \sigma_c(A) \Leftrightarrow dom (A - \lambda I)^{-1} \ne L_2[0,1], cl dom (A - \lambda I) ^ {-1} = L_2 [0,1]$

$(A-\lambda I)\varphi=\psi\Leftrightarrow q\varphi(q)-\lambda\varphi(q)=\psi(q)$

$\varphi(q)=\frac{\psi(q)}{q-\lambda}$

$ \lambda \notin [0, 1]; \exists \varepsilon > 0 : \forall q \in [0,1] |q-\lambda|\ge \varepsilon $

$ dom(A-\lambda I)^{-1}=\{\psi\in L_2[0,1]|\varphi=(A-\lambda I)^{-1},\psi \in L_2[0,1]\}$

$ ||\varphi||^2_{L_2} = \int_0^1|\varphi(q)|^2dq=\int_0^1\frac{|\psi|^2}{|q-\lambda|^2}dq \le \frac{1}{\varepsilon^2}\int_0^1|\psi|^2dq < +\infty$

$ \Rightarrow \lambda \notin [0,1] $ является регулярным значением оператора $A$, т.е. $dom(A-\lambda I)^{-1}=L_2[0,1]$

пусть $ \lambda \in [0,1]$, предположим, что $dom(A-\lambda I)^{-1}\ne L_2[0,1]$, т.е. $\exists \psi \in L_2 : \varphi = \frac{\psi}{q-\lambda}\notin L_2 $

$ \psi = 1 \in L_2 $

$ ||\varphi||^2 = \int _0^1 \frac{1}{q-\lambda}dq = \int_0^\lambda + \int_\lambda^1 = \frac{1}{q-\lambda}|_0^\lambda+\frac{1}{q-\lambda}|_\lambda^1=\infty \Rightarrow dom(A-\lambda i) \ne L_2 $

$ Z_\lambda = \{\psi \in L_2 | \exists \epsilon > 0 : \forall q \in [0,1] : |q - \lambda|\le \varepsilon, \psi(q) = 0\} $

утв. 1: $cl Z_\lambda = L_2 \Leftrightarrow ||\varphi_\epsilon - \varphi || \rightarrow _ {\epsilon \rightarrow 0} 0 \Leftrightarrow \varphi_ \epsilon \rightarrow _{\epsilon \rightarrow 0} 0$

утв. 2: $Z_\lambda \in dom(A-\lambda I)^{-1} \Rightarrow cl dom (A-\lambda I)^{-1} = L_2$

$\psi \in Z_\lambda$, надо доказать, что $\psi \in dom (A-\lambda I)^{-1}$, т.е. что $\varphi = (A-\lambda I)^{-1}\psi=\frac{\psi(q)}{q-\lambda} \in L_2$

$||\varphi||^2 = \int_0^1\frac{|\psi(q)|^2}{|q-\lambda|^2}dq = \int_0^{\lambda - \epsilon} + \int_{\lambda - \epsilon}^{\lambda + \epsilon} + \int_{\lambda + \epsilon}^1=\int_0^{\lambda - \epsilon}\frac{|\varphi(q)|^2}{|q-\lambda|^2}dq+\int_{\lambda + \epsilon}^1\frac{|\varphi(q)|^2}{|q-\lambda|^2}dq\le\frac{1}{\epsilon^2}\int_0^1|\varphi(q)|^2dq<\infty$

г) $\sigma_p = \varnothing \Rightarrow A$ не компактен

\end{document}
