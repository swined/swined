\documentclass[russian]{article}
\usepackage[T1]{fontenc}
\usepackage[utf8]{inputenc}
\usepackage{amsmath, amssymb}
\usepackage{esint}
\usepackage{babel}
\makeatletter
\makeatother

\begin{document}
2.1

а) $A(a\varphi_1+b\varphi_2)=(a\varphi_1+b\varphi_2)(x^\alpha)=a\varphi_1(x^\alpha)+b\varphi_2(x^\alpha)=aA\varphi_1+bA\varphi_2$

б) $A$ непрерывен как суперпозиция двух непрерывных функций

$||\varphi||_{C[0,1]}=\sup_{x\in[0,1]}|\varphi(x)|$

$||A\varphi||_{C[0,1]}=\sup_{x\in[0,1]}|\varphi(x^\alpha)|=_{t=x^\alpha}\sup_{t\in[0,1]}|\varphi(t)|=||\varphi||_{C[0,1]}$

в) $||A\varphi||^2=\int_0^1|\varphi(x^\alpha)|^2dx=_{t=x^\alpha}\frac{1}{\alpha}\int_0^1|\varphi(t)|^2t^{\frac{1}{\alpha}-1}dt$

$\frac{1}{\alpha}-1\ge0\Rightarrow\int_0^1|\varphi(t)|^2t^{\frac{1}{\alpha}-1}dt\le\int_0^1|\varphi(t)|^2dt<+\infty$

$\alpha>1\Rightarrow\varphi(x)=x^\beta\in L_2[0,1]\Leftrightarrow\int_0^1|x^\beta|^2dx=\int_0^1x^{2\beta}dx=\frac{x^{2\beta+1}}{2\beta+1}|_0^1=\frac{1}{2\beta+1}<_{2\beta+1>0}+\infty$

$A\varphi\in L_2[0,1] \Leftrightarrow \int_0^1|(x^\alpha)^\beta|^2dx=\int_0^1x^{2\alpha\beta}dx=\frac{x^{2\alpha\beta+1}}{2\alpha\beta+1}|_0^1<+\infty \Leftrightarrow 2\alpha\beta+1>0 \Rightarrow \beta > -\frac{1}{2\alpha}$

г) $ ||A||=\sup_{x\ne0}\frac{||Ax||}{||x||}=\sup_{x\ne0}\frac{\sqrt{\int_0^1|Ax(t)|^2dt}}{\sqrt{\int_0^1|x(t)|^2dt}}=\sup_{x\ne0}\frac{\sqrt{\int_0^1|x(t^\alpha)|^2dt}}{\sqrt{\int_0^1|x(t)|^2dt}}=\sup_{x\ne0}\frac{\sqrt{\int_0^1|x(u)|^2\frac{1}{\alpha}u^{\frac{1}{\alpha}-1}du}}{\sqrt{\int_0^1|x(t)|^2dt}} $

$ 0 < \alpha \le 1 \Rightarrow u^{\frac{1}{\alpha}-1} \le 1 \forall u \in [0,1] \Rightarrow ||A|| \le \frac{1}{\sqrt{\alpha}} \sup_{x\ne0}\frac{\sqrt{\int_0^1|x(u)|^2du}}{\sqrt{\int_0^1|x(t)|^2dt}}=\frac{1}{\sqrt{\alpha}}$

$ x(t)=t^\beta : ||A|| \ge \sup_{\beta}\frac{\sqrt{\int_0^1u^{2\beta}\frac{1}{\alpha}u^{\frac{1}{\alpha}-1}}}{\sqrt{\int_0^1t^{2\beta}dt}} = \frac{1}{\sqrt{\alpha}}\sup_{\beta}\frac{\sqrt{\int_0^1u^{2\beta+\frac{1}{\alpha}-1}}}{\sqrt{\int_0^1t^{2\beta}dt}}$

$ 2 \beta + 1 > 0 : \int_0^1t^{2\beta}dt = \frac{1}{2\beta+1} $

$ 2\beta + \frac{1}{\alpha} > 0 : \int_0^1u^{2\beta+\frac{1}{\alpha}-1} = \frac{1}{2\beta+\frac{1}{\alpha}} $

$ ||A|| \ge \frac{1}{\alpha} \sup _ \beta \frac{\sqrt{2\beta+1}}{\sqrt{2\beta+\frac{1}{\alpha}}} \ge_{\beta \rightarrow \infty } \frac{1}{\sqrt{\alpha}}$

$ ||A|| = \frac{1}{\sqrt{\alpha}} $

2.2

а) $ ||A\varphi||^2=\int_0^1|A\varphi(q)|^2dq=\int_{0}^{1}|q\varphi(q)|^2dq \le \int{}_{0}^{1}|\varphi (q)|^2dq =|| \varphi ||^2<+\infty$ т.к. $\varphi \in L_2 $

$ ||A|| = \sup _{\varphi \ne 0 } \frac{||A\varphi||}{||\varphi||} \le 1 $ 

$ ||A|| = \sup _{\varphi \ne 0 } \frac{||A\varphi||}{||\varphi||} \ge _{\varphi=q^\beta} \sup_\beta \frac{\sqrt{\int_0^1q^2q^{2\beta}dq}}{\sqrt{\int_0^1q^{2\beta}dq}} = \sup _\beta \frac{\sqrt{2\beta + 1}}{\sqrt{2\beta+3}} \ge _{\beta \rightarrow \infty} 1$

$ ||A|| = 1 $

б) $q\varphi(q)-\lambda\varphi(q)=0\Rightarrow(q-\lambda)\varphi(q)=0\Rightarrow\varphi=_{L_{2}}0 \Rightarrow $ ненулевых решений нет $\Rightarrow$ $\sigma_{p}(A)=\varnothing$

в) $ \lambda \in \sigma_c(A) \Leftrightarrow dom (A - \lambda I)^{-1} \ne L_2[0,1], cl dom (A - \lambda I) ^ {-1} = L_2 [0,1]$

$(A-\lambda I)\varphi=\psi\Leftrightarrow q\varphi(q)-\lambda\varphi(q)=\psi(q)$

$\varphi(q)=\frac{\psi(q)}{q-\lambda}$

$ \lambda \notin [0, 1]; \exists \varepsilon > 0 : \forall q \in [0,1] |q-\lambda|\ge \varepsilon $

$ dom(A-\lambda I)^{-1}=\{\psi\in L_2[0,1]|\varphi=(A-\lambda I)^{-1},\psi \in L_2[0,1]\}$

$ ||\varphi||^2_{L_2} = \int_0^1|\varphi(q)|^2dq=\int_0^1\frac{|\psi|^2}{|q-\lambda|^2}dq \le \frac{1}{\varepsilon^2}\int_0^1|\psi|^2dq < +\infty$

$ \Rightarrow \lambda \notin [0,1] $ является регулярным значением оператора $A$, т.е. $dom(A-\lambda I)^{-1}=L_2[0,1]$

пусть $ \lambda \in [0,1]$, предположим, что $dom(A-\lambda I)^{-1}\ne L_2[0,1]$, т.е. $\exists \psi \in L_2 : \varphi = \frac{\psi}{q-\lambda}\notin L_2 $

$ \psi = 1 \in L_2 $

$ ||\varphi||^2 = \int _0^1 \frac{1}{q-\lambda}dq = \int_0^\lambda + \int_\lambda^1 = \frac{1}{q-\lambda}|_0^\lambda+\frac{1}{q-\lambda}|_\lambda^1=\infty \Rightarrow dom(A-\lambda i) \ne L_2 $

$ Z_\lambda = \{\psi \in L_2 | \exists \epsilon > 0 : \forall q \in [0,1] : |q - \lambda|\le \varepsilon, \psi(q) = 0\} $

утв. 1: $cl Z_\lambda = L_2 \Leftrightarrow ||\varphi_\epsilon - \varphi || \rightarrow _ {\epsilon \rightarrow 0} 0 \Leftrightarrow \varphi_ \epsilon \rightarrow _{\epsilon \rightarrow 0} 0$

утв. 2: $Z_\lambda \in dom(A-\lambda I)^{-1} \Rightarrow cl dom (A-\lambda I)^{-1} = L_2$

$\psi \in Z_\lambda$, надо доказать, что $\psi \in dom (A-\lambda I)^{-1}$, т.е. что $\varphi = (A-\lambda I)^{-1}\psi=\frac{\psi(q)}{q-\lambda} \in L_2$

$||\varphi||^2 = \int_0^1\frac{|\psi(q)|^2}{|q-\lambda|^2}dq = \int_0^{\lambda - \epsilon} + \int_{\lambda - \epsilon}^{\lambda + \epsilon} + \int_{\lambda + \epsilon}^1=\int_0^{\lambda - \epsilon}\frac{|\varphi(q)|^2}{|q-\lambda|^2}dq+\int_{\lambda + \epsilon}^1\frac{|\varphi(q)|^2}{|q-\lambda|^2}dq\le\frac{1}{\epsilon^2}\int_0^1|\varphi(q)|^2dq<\infty$

$\sigma_c = [0,1]$

г) $\sigma_p = \varnothing \Rightarrow A$ не компактен

2.3

а) $A(x_1, ..., x_n) = (a_1 x_1, ..., a_n x_n)$

$A(\alpha x + \beta y) = (\alpha a_1 x_1 + \beta a_1 y_1, ..., \alpha a_n x_n + \beta a_n y_n) = \alpha (a_1 x_1, ..., a_n x_n) + \beta (a_1 y_1, ..., a_n, y_n) = \alpha A x + \beta A y$

б) $l_2 = \{ (x_n) | \sum_{n=1}^{\infty}|x_n|^2<\infty \}$

$\forall x \in l_2 : Ax \in l_2 $

доказать: если $(a_n)$ ограничена ($\exists c < \infty : \forall n : a_n \le c $), то $\sum_1^\infty |x_n|^2 < \infty \Rightarrow \sum_1^\infty  |a_n x_n| ^2 < \infty$

1) $ \sum_1^\infty  |a_n x_n| ^2 \le \sum_1^\infty  |c x_n| ^2 = c^2 \sum_1^\infty |x_n| ^2 < \infty$

2) допустим $(a_n)$ не ограничена, т.е. $\forall N \in \Bbb N \exists n(N) : a_{n(N)} > N, \forall i>j : a_i>a_j $

$x_i = \{ \forall i \in (n(N)) : \frac{1}{a_{n(N)}} | \forall i \notin (n(N)) : 0 \}$

$ \sum _1^\infty (x_n)^2 = \sum _1^\infty \frac{1}{a_{n(N)}^2} < \sum _1^\infty \frac{1}{N^2} = \frac{\pi^2}{6} < \infty $

$ \sum _1^\infty (a_n x_n)^2 = \sum _{n\in(n(N))} (a_n x_n)^2 = \sum _{N=1}^\infty (a_{n(N)} x_{n(N)})^2 = \sum _{N=1}^\infty (\frac{a_{n(N)}}{a_{n(N)}})^2 = \sum 1 = \infty $

2.4 а) $||x||=\sqrt{\sum_n=1^\infty|x_n|^2}$

$(a_n)$ ограничена $ \Rightarrow \exists c < \infty : \forall n : a_n \le c $ 

$||Ax||=\sqrt{\sum_n=1^\infty|a_n x_n|^2}=c\sqrt{\sum_n=1^\infty|x_n|^2}=c||x||$

$||A||=\sup_{x \ne 0} \frac{||Ax||}{||x||} \le c < \infty \Leftrightarrow A $ непрерывен

$||A||=\sup_{x \ne 0} \frac{||Ax||}{||x||} \ge \sup_{e_n=(0, 0, ..., e_n = 1, 0, ...)} \frac{Ae_n}{e_n}=\sup_n \frac{||(0, 0, ..., a_n, 0, ...)||}{||(0, 0, ..., 1, 0, ...)||}=\sup_n |a_n| \Rightarrow ||A||=\sup_n |a_n|$

б) $(\Leftarrow) $ $A$ компактен, если $\exists A_n : l_2 \rightarrow l_2$ : 

1) $\forall n : A_n$ непрерывен

2) $\forall n : dim(im(A_n)) < \infty$

3) $A_n \rightarrow_{n \rightarrow \infty} A$ ($||A_n - A||\rightarrow_{n \rightarrow \infty} 0$)

построим $A_n:l_2\rightarrow l_2$ : 

$A_n (x_1, x_2, ..., x_n, ...)=(a_1 x_1, a_2 x_2, ..., a_n x_n, 0, 0, ...)$

1) $A_n$ непр. по п. (а)

2) $im(A_n)=\{y\in l_2 | \exists x \in l_2 : y = Ax\} \subset \{y \in l_2 | (y_1, y_2, ..., y_n, 0, 0, ...)\}$

$dim(A_n) = n \Rightarrow dim(im(A_n)) \le dim(A_n) < \infty$

3) $||A_n - A|| \le \sup _{k \ge 1} |a_{n+k}|$

$a_n \rightarrow_{n\rightarrow \infty} 0 \Rightarrow ||A_n - A|| \rightarrow_{n \rightarrow \infty} 0$

в) $Ax = \lambda x$

$(a_1 x_1, a_2 x_2, ..., a_n x_n, ...) = (\lambda x_1, \lambda x_2, ..., \lambda x_n, ...)$

$\forall n:\lambda \ne a_n \Rightarrow \forall n : x_n = 0 \Rightarrow \lambda \notin \sigma_p(A)$

$\forall n:\lambda = a_n \Rightarrow (0, 0, ..., x_n = 1, 0, ...)$ - ненулевое решение $\Rightarrow \sigma_p = \{a_n\}$

$\lambda \notin \sigma_p(A) \Rightarrow \exists (A-\lambda I)^{-1} : (A-\lambda I)^{-1}(y) = (\frac{y_1}{a_1-\lambda}, \frac{y_2}{a_2-\lambda}, ..., \frac{y_n}{a_n-\lambda}, ...)$

если $dom(A-\lambda I)^{-1}=l_2$, то $\lambda \in \rho(A) = \Bbb{C} \setminus \sigma(A)$

(а) $ \Rightarrow dom(A-\lambda I)^{-1}=l_2$ при $ \sup_n |\frac{1}{a_n-\lambda}|<C \Rightarrow \forall n : |a_n - \lambda| \ge \delta \Rightarrow \lambda$ не является предельной точкой $(a_n) \Rightarrow \sigma \subset cl\{a_n\}$

Т.: спектр - замкнутое множество $ \Rightarrow cl\{a_n\} \subset \sigma$

$\sigma \subset cl\{a_n\} \subset \sigma \Rightarrow \sigma = cl\{a_n\}$

$\forall \lambda \in Lim\{a_n\} : dom(A-\lambda I)^{-1} \ne l_2$ т.к. $\lambda \in \sigma$

$(A-\lambda I)^{-1}(y) = (\frac{y_1}{a_1-\lambda}, \frac{y_2}{a_2-\lambda}, ..., \frac{y_n}{a_n-\lambda}, ...)$

$dom(A-\lambda I)^{-1}=\{y \in l_2|(A-\lambda I)^{-1}y \in l_2\}$

$L_n = \{y \in l_2 | (y_1, ..., y_n, 0 , ...)\}$

$\forall n : L_n \subset dom(A-\lambda I)^{-1} \Rightarrow \sum _n=1 ^\infty L_n \subset dom(A- \lambda I)^{-1}$

$\forall x \in l_2 : x=(x_1, x_2, ..., x_n, ...)$ рассмотрим $X_n = (x_1, x_2, ..., x_n, 0, 0, ...) \in L_n$

$||x-X_n|| = \sqrt{\sum_{k=1}^\infty |x_{n+k}|^2}\rightarrow_{n\rightarrow \infty} 0$ т.к. это хвост сходящегося ряда $\sum|x_n|^2 \Rightarrow cl(\sum L_n)=l_2 \Rightarrow cl(dom(A-\lambda I)^{-1})=l_2 \Rightarrow \sigma_c = Lim \{a_n\} \Rightarrow \sigma_r = \varnothing$

б) $(\Rightarrow) $

Т.: если $A$ компактен, то $\forall \epsilon > 0 \exists $ лишь конечное число его собственных значений таких, что $|\lambda| \ge \epsilon$

предположим $a_n \to\!\!\!\!\!\!/\ \ 0$ при $n \to \infty \Leftrightarrow \exists \epsilon > 0 \forall n_0 \exists n \ge n_0, a_n \ge \epsilon$ т.е. $\exists $ бесконечно много $n$ таких, что $|a_n| \ge \epsilon$, а это противоречит теореме

г) $(Ax, y) = (x, By)$

$(Ax, y) = (a_1 x_1 \overline{y_1} + a_2 x_2 \overline{y_2} + ... + a_n x_n \overline{y_n} + ...) = (x_1 \overline{\overline{a_1}y_1} + x_2 \overline{\overline{a_2}y_2} + ... + x_n \overline{\overline{a_n}y_n} + ...) = (x, By)$

где $By = (\overline{a_1}y_1, \overline{a_2}y_2, ..., \overline{a_n}y_n, ...)$

д) $A^* = A \Leftrightarrow \forall n : a_n = \overline{a_n} \Leftrightarrow \forall n : a_n \in \Bbb R$

е) $A$ унитарен, если $(Ax, Ay) = (x, y) \forall x,y \in l_2$

$(Ax, Ay) = ((a_1 x_1, a_2 x_2, ...), (a_1 y_1, a_2 y_2, ...)) = a_1 x_1 \overline{a_1} \overline{y_1} + a_2 x_2 \overline{a_2} \overline{y_2} + ... $

$\forall n : a_n \overline{a_n} = 1$

2.5 а) $(I\varphi, \psi) = \int_0^1(I\varphi)(x)\overline{\psi(x)}dx=\int_0^1[\int_0^x\varphi(t)dt]\overline{\psi(x)}dx=\iint \varphi(t) \overline{\psi(x)} dt dx = \int_0^1[\int_t^1\overline{\psi(x)}dx]\varphi(t)dt = \int_0^1\varphi(t)\overline{[\int_t^1\psi(x)dx]}dt = (\varphi, I^* \psi) \Rightarrow (I^* \psi) (t) = \int_t^1\psi(x)dx$

$\int_0^1[\int_0^1|\varphi(t)\psi(x)|^2dt]dx=\int_0^1[\int_0^1|\varphi(t)|^2 dt]|\psi(x)|^2 dx = \int|\varphi(t)|^2 dt \int_0^1 |\psi(x)|^2 dx < \infty \Rightarrow \iint | \varphi(t) \psi(x)|^2 dt dx < \infty \Rightarrow \iint \varphi(t) \psi(x) dt dx < \infty $

б) $(E_h \varphi, \psi) = \int_0^1(E_h \varphi)(x)\overline{\psi (x)} dx = \int_0^h\varphi (x) \overline{\psi(x)} dx = \int_0^h\varphi (x) \overline{E_h \psi(x)} dx = \int_0^1\varphi (x) \overline{E_h \psi(x)} dx \Rightarrow E_h = E_h^*$

в) $(A_\alpha \varphi, \psi) = \int_0^1\varphi(x^\alpha)\overline{\psi(x)}dx = _{t=x^\alpha} \int_0^1\varphi(t)\overline{\frac{1}{\alpha} t^{\frac{1}{\alpha}-1}\psi(t^\frac{1}{\alpha})}dt \Rightarrow A_\alpha^* \psi (t)= \frac{1}{\alpha} t^{\frac{1}{\alpha}-1}\psi(t^\frac{1}{\alpha})$

г) $E_h \varphi = \lambda \varphi$

$h \le 0$ : $\lambda = 0 \in \sigma_c$, $\varphi$ - любая

$0 < h < 1$ : $\lambda = 1 \in \sigma_c$, $\varphi$ - любая при $x < h$, $\varphi = 0$ при $x > h$

$h > 1$ : $\lambda = 1$, $\varphi$ - любая

$(E_h - \lambda I) \varphi = \psi$

$x < h$ : $\varphi (x) - \lambda \varphi(x) = \psi (x) \Rightarrow \varphi (x) = \frac{\psi (x)}{1 - \lambda}$

$x > h$ : $-\lambda \varphi (x) = \psi \Rightarrow \varphi (x) = \frac{\psi (x)}{ - \lambda}$

$dom(E_h^*) = L_2 \Rightarrow \rho = \Bbb C \setminus \sigma_p \Rightarrow \sigma_c = \sigma_r = \varnothing$

2.6 Т.: $\sigma (A) \subset \{ \lambda \in \Bbb C | |\lambda| \le ||A||\}$

T.: $cl(\sigma) = \sigma$

$\Rightarrow \sigma $ - ограниченное замкнутое множество на плоскости

$\Leftarrow F $ - ограниченное замкнутое множество на плоскости

построим оператор $A$ такой, что $\sigma(A) = F$

$A:l_2 \to l_2$

$A(x_1, x_2, ..., x_n, ...) = (a_1 x_1, a_2 x_2, ..., a_n x_n, ...)$, $a_n$ всюду плотно в $F$

тогда $\sigma A = F$

т.е. подмножество комплексной плоскости является спектром некоторого оператора, если и только если оно замкнуто и ограничено

2.7 а) $\varphi (1) = 0$

$A \varphi = \lambda \varphi$

$\frac{d\varphi}{dx} = i\lambda \varphi \Rightarrow \varphi(t) = C e^{i\lambda t}$. т.к. $\varphi (1) = 0$, то $\sigma_p = \varnothing$

$(A- \lambda I)\varphi = \psi$

$\frac{1}{i}\frac{d\varphi}{dt} - \lambda \varphi(t) = \psi(t)$

$\varphi'-i \lambda \varphi = i \psi$

$\varphi = C e^{i \lambda t} + C(t)e^{i \lambda t}$

$C'e^{i\lambda t} + C(i \lambda) e ^ {i \lambda t} - C(i \lambda) e ^ {i \lambda t} = i \psi$

$C'e^{i\lambda t} = i \psi$

$C' = \frac{dC}{dt}=i\psi e^{-i \lambda t}$

$C(t) = \int _1^t i\psi(s)e^{-i\lambda s}ds + C_1$

$\varphi(t) = Ce^{i \lambda t} + e ^ {i \lambda t} [ \int _1^t i\psi(s) e ^{-i \lambda s}ds + C_1]$

$\varphi(1) = 0 \Rightarrow e^{i \lambda} [C + \int_1^1 + C_1] = 0 \Rightarrow C_1 = -C \Rightarrow \varphi(t) = ie^{i\lambda t} \int_1^t\psi(s)e^{-i 
\lambda s} ds$

б) $2\varphi(0)+\varphi(1) = 0$

$2\varphi(0)+\varphi(1) = 0 \Rightarrow 2 C e^{i \lambda 0} + C e^{i \lambda 1} = 0 \Rightarrow 2 C + C e^{i \lambda} = 0 \Rightarrow e^{i \lambda} = -2 \Rightarrow \lambda = -i \ln 2 + 2\pi k$

$\sigma_p = \{-i \ln 2 + 2\pi k \forall k \in \Bbb N\}$

$\varphi(t) = Ce^{i \lambda t} + e ^ {i \lambda t} [ \int _1^t i\psi(s) e ^{-i \lambda s}ds + C_1] = (C + C_1)e^{i \lambda t} + e ^ {i \lambda t} \int _1^t i\psi(s) e ^{-i \lambda s}ds$

$0 = 2\varphi(0)+\varphi(1) = 2 C + 2 [ \int _1^0 i\psi(s) e ^{-i \lambda s}ds + C_1] + Ce^{i \lambda} + e ^ {i \lambda} [ \int _1^1 + C_1] = (C + C_1)(2+e ^ {i \lambda}) - 2 \int _0^1 i\psi(s) e ^{-i \lambda s}ds \Rightarrow C + C_1 = \frac{2}{2+e ^ {i \lambda}} \int _0^1 i\psi(s) e ^{-i \lambda s}ds \Rightarrow \varphi(t) = e^{i \lambda t} (\frac{2}{2+e ^ {i \lambda}} \int _0^1 i\psi(s) e ^{-i \lambda s}ds + \int _1^t i\psi(s) e ^{-i \lambda s}ds) \Rightarrow \sigma_c = \varnothing$

неравенство Коши-Буняковского: $\int_a^b |f(x) g(x)|^2 \le \int _a^b |f(x)|^2 dx \int _a^b |g(x)|^2dx$

2.8 $A\varphi = \lambda \varphi$

$\int_0^1(s t - s^2 t^2) \varphi (t) dt = \lambda \varphi (s)$

$s\int_0^1t\varphi(t)dt - s^2 s\int_0^1t^2\varphi(t)dt = \lambda \varphi (s)$

$\lambda \varphi(s) = a s - b s^2$

1) $\lambda \ne 0 \Rightarrow \varphi(s) = A s - B s^2$

$\int_0^1(s t - s^2 t^2)(A t - B t^2) dt = \lambda (A s - B s^2)$

$s \int _0^1 t(A t - B t^2) dt - s^2\int_0^1t^2(A t - B t^2) dt$

$s[A \frac{1}{3} - B \frac {1}{4}] - s^2 [A\frac{1}{4} - B\frac{1}{5}] = \lambda A s - \lambda B s ^2$

$A \frac{1}{3} - B \frac {1}{4} = \lambda A$, $A\frac{1}{4} - B\frac{1}{5} = \lambda B$

\begin{equation*}
\begin{pmatrix}
\frac{1}{3}-\lambda & -\frac{1}{4} \\
\frac{1}{4} & -\frac{1}{5}-\lambda
\end{pmatrix} 
\begin{pmatrix}
A \\
B
\end{pmatrix}
=
\begin{pmatrix}
0 \\
0
\end{pmatrix}
\end{equation*}

ненулевые решения получатся только если

$$
det
\begin{pmatrix}
\frac{1}{3}-\lambda & -\frac{1}{4} \\
\frac{1}{4} & -\frac{1}{5}-\lambda
\end{pmatrix} 
=0
$$

$15 \lambda ^2 - 2 \lambda - \frac{1}{16} = 0$

$d = 4 + 4 \frac{15}{16} = \frac{31}{4}$

$\lambda = \frac{4 \pm \sqrt{31}}{60} \in \sigma_p$

2) $\lambda = 0 \Rightarrow s\int_0^1t\varphi(t)dt - s^2 \int _0 ^1 t^2 \varphi (t) dt = 0 \Rightarrow \lambda = 0 \in \sigma _p$

$(A-\lambda I)\varphi = \psi$

$s\int_0^1t\varphi(t)dt - s^2 \int_0^1t^2\varphi(t)dt - \lambda \varphi(s)= \psi(s)$

$s a - s^2 b - \lambda \varphi(s)= \psi(s)$

$\lambda \varphi(s) = s a - s^2 b - \psi(s)$

$\varphi(s) = s \frac{a}{\lambda} - s^2 \frac{b}{\lambda} - \frac{\psi(s)}{\lambda}$

$s\int_0^1t(\frac{a}{\lambda} t + \frac{b}{\lambda}t^2 - \frac{\psi(t)}{\lambda})dt - s^2 \int _0^1 t^2(\frac{a}{\lambda} t + \frac{b}{\lambda}t^2 - \frac{\psi(t)}{\lambda})dt - (a s + b s^2 - \psi(s)) = \psi(s)$

$s\int_0^1t(\frac{a}{\lambda} t + \frac{b}{\lambda}t^2 - \frac{\psi(t)}{\lambda})dt - s^2 \int _0^1 t^2(\frac{a}{\lambda} t + \frac{b}{\lambda}t^2 - \frac{\psi(t)}{\lambda})dt - (a s + b s^2) = 0$

$s[\frac{a}{\lambda}\frac{1}{3}-\frac{b}{\lambda}\frac{1}{4} - \frac{1}{\lambda}\int_0^1t\psi(t)dt] - s^2[\frac{a}{\lambda}\frac{1}{4}-\frac{b}{\lambda}\frac{1}{5} - \frac{1}{\lambda}\int_0^1 t^2\psi(t) dt] - (a s - b s ^ 2) = 0$

$\frac{a}{\lambda}\frac{1}{3} - a - \frac{b}{\lambda}\frac{1}{4} = \frac{1}{\lambda}\int_0^1 t \psi (t) dt$

$\frac{1}{\lambda}\frac{1}{4} - \frac{b}{\lambda}\frac{1}{5} = \frac{1}{\lambda}\int_0^1t^2 \psi(t) dt$

\begin{equation*}
\begin{pmatrix}
\frac{1}{3}-\lambda & -\frac{1}{4} \\
\frac{1}{4} & -\frac{1}{5}-\lambda
\end{pmatrix} 
\begin{pmatrix}
A \\
B
\end{pmatrix}
=
\begin{pmatrix}
\frac{1}{\lambda}\int_0^1 t \psi (t) dt \\
\frac{1}{\lambda}\int_0^1 t^2 \psi(t) dt
\end{pmatrix}
\end{equation*}

при $det \ne 0$ существует единственное решение $a,b$ и соответственно $\varphi \in L_2$, т.е. $\rho = \Bbb C \setminus \sigma_p \Rightarrow \sigma_c = \sigma_r = \varnothing$

\end{document}

