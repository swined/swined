\documentclass[russian]{article}
\usepackage[T1]{fontenc}
\usepackage[utf8]{inputenc}
\usepackage{amssymb}
\usepackage{esint}
\usepackage{babel}
\makeatletter
\makeatother

\begin{document}
2.1

а) $A(a\varphi_1+b\varphi_2)=(a\varphi_1+b\varphi_2)(x^\alpha)=a\varphi_1(x^\alpha)+b\varphi_2(x^\alpha)=aA\varphi_1+bA\varphi_2$

б) $A$ непрерывен как суперпозиция двух непрерывных функций

$||\varphi||_{C[0,1]}=\sup_{x\in[0,1]}|\varphi(x)|$

$||A\varphi||_{C[0,1]}=\sup_{x\in[0,1]}|\varphi(x^\alpha)|=_{t=x^\alpha}\sup_{t\in[0,1]}|\varphi(t)|=||\varphi||_{C[0,1]}$

в) $||A\varphi||^2=\int_0^1|\varphi(x^\alpha)|^2dx=_{t=x^\alpha}\frac{1}{\alpha}\int_0^1|\varphi(t)|^2t^{\frac{1}{\alpha}-1}dt$

$\frac{1}{\alpha}-1\ge0\Rightarrow\int_0^1|\varphi(t)|^2t^{\frac{1}{\alpha}-1}dt\le\int_0^1|\varphi(t)|^2dt<+\infty$

$\alpha>1\Rightarrow\varphi(x)=x^\beta\in L_2[0,1]\Leftrightarrow\int_0^1|x^\beta|^2dx=\int_0^1x^{2\beta}dx=\frac{x^{2\beta+1}}{2\beta+1}|_0^1=\frac{1}{2\beta+1}<_{2\beta+1>0}+\infty$

$A\varphi\in L_2[0,1] \Leftrightarrow \int_0^1|(x^\alpha)^\beta|^2dx=\int_0^1x^{2\alpha\beta}dx=\frac{x^{2\alpha\beta+1}}{2\alpha\beta+1}|_0^1<+\infty \Leftrightarrow 2\alpha\beta+1>0 \Rightarrow \beta > -\frac{1}{2\alpha}$

г) $ ||A||=\sup_{x\ne0}\frac{||Ax||}{||x||}=\sup_{x\ne0}\frac{\sqrt{\int_0^1|Ax(t)|^2dt}}{\sqrt{\int_0^1|x(t)|^2dt}}=\sup_{x\ne0}\frac{\sqrt{\int_0^1|x(t^\alpha)|^2dt}}{\sqrt{\int_0^1|x(t)|^2dt}}=\sup_{x\ne0}\frac{\sqrt{\int_0^1|x(u)|^2\frac{1}{\alpha}u^{\frac{1}{\alpha}-1}du}}{\sqrt{\int_0^1|x(t)|^2dt}} $

$ 0 < \alpha \le 1 \Rightarrow u^{\frac{1}{\alpha}-1} \le 1 \forall u \in [0,1] \Rightarrow ||A|| \le \frac{1}{\sqrt{\alpha}} \sup_{x\ne0}\frac{\sqrt{\int_0^1|x(u)|^2du}}{\sqrt{\int_0^1|x(t)|^2dt}}=\frac{1}{\sqrt{\alpha}}$

$ x(t)=t^\beta : ||A|| \ge \sup_{\beta}\frac{\sqrt{\int_0^1u^{2\beta}\frac{1}{\alpha}u^{\frac{1}{\alpha}-1}}}{\sqrt{\int_0^1t^{2\beta}dt}} = \frac{1}{\sqrt{\alpha}}\sup_{\beta}\frac{\sqrt{\int_0^1u^{2\beta+\frac{1}{\alpha}-1}}}{\sqrt{\int_0^1t^{2\beta}dt}}$

$ 2 \beta + 1 > 0 : \int_0^1t^{2\beta}dt = \frac{1}{2\beta+1} $

$ 2\beta + \frac{1}{\alpha} > 0 : \int_0^1u^{2\beta+\frac{1}{\alpha}-1} = \frac{1}{2\beta+\frac{1}{\alpha}} $

$ ||A|| \ge \frac{1}{\alpha} \sup _ \beta \frac{\sqrt{2\beta+1}}{\sqrt{2\beta+\frac{1}{\alpha}}} \ge_{\beta \rightarrow \infty } \frac{1}{\sqrt{\alpha}}$

$ ||A|| = \frac{1}{\sqrt{\alpha}} $

2.2

а) $ ||A\varphi||^2=\int_0^1|A\varphi(q)|^2dq=\int_{0}^{1}|q\varphi(q)|^2dq \le \int{}_{0}^{1}|\varphi (q)|^2dq =|| \varphi ||^2<+\infty$ т.к. $\varphi \in L_2 $

$ ||A|| = \sup _{\varphi \ne 0 } \frac{||A\varphi||}{||\varphi||} \le 1 $ 

$ ||A|| = \sup _{\varphi \ne 0 } \frac{||A\varphi||}{||\varphi||} \ge _{\varphi=q^\beta} \sup_\beta \frac{\sqrt{\int_0^1q^2q^{2\beta}dq}}{\sqrt{\int_0^1q^{2\beta}dq}} = \sup _\beta \frac{\sqrt{2\beta + 1}}{\sqrt{2\beta+3}} \ge _{\beta \rightarrow \infty} 1$

$ ||A|| = 1 $

б) $q\varphi(q)-\lambda\varphi(q)=0\Rightarrow(q-\lambda)\varphi(q)=0\Rightarrow\varphi=_{L_{2}}0 \Rightarrow $ ненулевых решений нет $\Rightarrow$ $\sigma_{p}(A)=\varnothing$

в) $ \lambda \in \sigma_c(A) \Leftrightarrow dom (A - \lambda I)^{-1} \ne L_2[0,1], cl dom (A - \lambda I) ^ {-1} = L_2 [0,1]$

$(A-\lambda I)\varphi=\psi\Leftrightarrow q\varphi(q)-\lambda\varphi(q)=\psi(q)$

$\varphi(q)=\frac{\psi(q)}{q-\lambda}$

$ \lambda \notin [0, 1]; \exists \varepsilon > 0 : \forall q \in [0,1] |q-\lambda|\ge \varepsilon $

$ dom(A-\lambda I)^{-1}=\{\psi\in L_2[0,1]|\varphi=(A-\lambda I)^{-1},\psi \in L_2[0,1]\}$

$ ||\varphi||^2_{L_2} = \int_0^1|\varphi(q)|^2dq=\int_0^1\frac{|\psi|^2}{|q-\lambda|^2}dq \le \frac{1}{\varepsilon^2}\int_0^1|\psi|^2dq < +\infty$

$ \Rightarrow \lambda \notin [0,1] $ является регулярным значением оператора $A$, т.е. $dom(A-\lambda I)^{-1}=L_2[0,1]$

пусть $ \lambda \in [0,1]$, предположим, что $dom(A-\lambda I)^{-1}\ne L_2[0,1]$, т.е. $\exists \psi \in L_2 : \varphi = \frac{\psi}{q-\lambda}\notin L_2 $

$ \psi = 1 \in L_2 $

$ ||\varphi||^2 = \int _0^1 \frac{1}{q-\lambda}dq = \int_0^\lambda + \int_\lambda^1 = \frac{1}{q-\lambda}|_0^\lambda+\frac{1}{q-\lambda}|_\lambda^1=\infty \Rightarrow dom(A-\lambda i) \ne L_2 $

$ Z_\lambda = \{\psi \in L_2 | \exists \epsilon > 0 : \forall q \in [0,1] : |q - \lambda|\le \varepsilon, \psi(q) = 0\} $

утв. 1: $cl Z_\lambda = L_2 \Leftrightarrow ||\varphi_\epsilon - \varphi || \rightarrow _ {\epsilon \rightarrow 0} 0 \Leftrightarrow \varphi_ \epsilon \rightarrow _{\epsilon \rightarrow 0} 0$

утв. 2: $Z_\lambda \in dom(A-\lambda I)^{-1} \Rightarrow cl dom (A-\lambda I)^{-1} = L_2$

$\psi \in Z_\lambda$, надо доказать, что $\psi \in dom (A-\lambda I)^{-1}$, т.е. что $\varphi = (A-\lambda I)^{-1}\psi=\frac{\psi(q)}{q-\lambda} \in L_2$

$||\varphi||^2 = \int_0^1\frac{|\psi(q)|^2}{|q-\lambda|^2}dq = \int_0^{\lambda - \epsilon} + \int_{\lambda - \epsilon}^{\lambda + \epsilon} + \int_{\lambda + \epsilon}^1=\int_0^{\lambda - \epsilon}\frac{|\varphi(q)|^2}{|q-\lambda|^2}dq+\int_{\lambda + \epsilon}^1\frac{|\varphi(q)|^2}{|q-\lambda|^2}dq\le\frac{1}{\epsilon^2}\int_0^1|\varphi(q)|^2dq<\infty$

$\sigma_c = [0,1]$

г) $\sigma_p = \varnothing \Rightarrow A$ не компактен

2.3

а) $A(x_1, ..., x_n) = (a_1 x_1, ..., a_n x_n)$

$A(\alpha x + \beta y) = (\alpha a_1 x_1 + \beta a_1 y_1, ..., \alpha a_n x_n + \beta a_n y_n) = \alpha (a_1 x_1, ..., a_n x_n) + \beta (a_1 y_1, ..., a_n, y_n) = \alpha A x + \beta A y$

б) $l_2 = \{ (x_n) | \sum_{n=1}^{\infty}|x_n|^2<\infty \}$

$\forall x \in l_2 : Ax \in l_2 $

доказать: если $(a_n)$ ограничена ($\exists c < \infty : \forall n : a_n \le c $), то $\sum_1^\infty |x_n|^2 < \infty \Rightarrow \sum_1^\infty  |a_n x_n| ^2 < \infty$

1) $ \sum_1^\infty  |a_n x_n| ^2 \le \sum_1^\infty  |c x_n| ^2 = c^2 \sum_1^\infty |x_n| ^2 < \infty$

2) допустим $(a_n)$ не ограничена, т.е. $\forall N \in \Bbb N \exists n(N) : a_{n(N)} > N, \forall i>j : a_i>a_j $

$x_i = \{ \forall i \in (n(N)) : \frac{1}{a_{n(N)}} | \forall i \notin (n(N)) : 0 \}$

$ \sum _1^\infty (x_n)^2 = \sum _1^\infty \frac{1}{a_{n(N)}^2} < \sum _1^\infty \frac{1}{N^2} = \frac{\pi^2}{6} < \infty $

$ \sum _1^\infty (a_n x_n)^2 = \sum _{n\in(n(N))} (a_n x_n)^2 = \sum _{N=1}^\infty (a_{n(N)} x_{n(N)})^2 = \sum _{N=1}^\infty (\frac{a_{n(N)}}{a_{n(N)}})^2 = \sum 1 = \infty $

2.4 а) $||x||=\sqrt{\sum_n=1^\infty|x_n|^2}$

$(a_n)$ ограничена $ \Rightarrow \exists c < \infty : \forall n : a_n \le c $ 

$||Ax||=\sqrt{\sum_n=1^\infty|a_n x_n|^2}=c\sqrt{\sum_n=1^\infty|x_n|^2}=c||x||$

$||A||=\sup_{x \ne 0} \frac{||Ax||}{||x||} \le c < \infty \Leftrightarrow A $ непрерывен

$||A||=\sup_{x \ne 0} \frac{||Ax||}{||x||} \ge \sup_{e_n=(0, 0, ..., e_n = 1, 0, ...)} \frac{Ae_n}{e_n}=\sup_n \frac{||(0, 0, ..., a_n, 0, ...)||}{||(0, 0, ..., 1, 0, ...)||}=\sup_n |a_n| \Rightarrow ||A||=\sup_n |a_n|$

б) $(\Leftarrow) $ $A$ компактен, если $\exists A_n : l_2 \rightarrow l_2$ : 

1) $\forall n : A_n$ непрерывен

2) $\forall n : dim(im(A_n)) < \infty$

3) $A_n \rightarrow_{n \rightarrow \infty} A$ ($||A_n - A||\rightarrow_{n \rightarrow \infty} 0$)

построим $A_n:l_2\rightarrow l_2$ : 

$A_n (x_1, x_2, ..., x_n, ...)=(a_1 x_1, a_2 x_2, ..., a_n x_n, 0, 0, ...)$

1) $A_n$ непр. по п. (а)

2) $im(A_n)=\{y\in l_2 | \exists x \in l_2 : y = Ax\} \subset \{y \in l_2 | (y_1, y_2, ..., y_n, 0, 0, ...)\}$

$dim(A_n) = n \Rightarrow dim(im(A_n)) \le dim(A_n) < \infty$

3) $||A_n - A|| \le \sup _{k \ge 1} |a_{n+k}|$

$a_n \rightarrow_{n\rightarrow \infty} 0 \Rightarrow ||A_n - A|| \rightarrow_{n \rightarrow \infty} 0$

в) $Ax = \lambda x$

$(a_1 x_1, a_2 x_2, ..., a_n x_n, ...) = (\lambda x_1, \lambda x_2, ..., \lambda x_n, ...)$

$\forall n:\lambda \ne a_n \Rightarrow \forall n : x_n = 0 \Rightarrow \lambda \notin \sigma_p(A)$

$\forall n:\lambda = a_n \Rightarrow (0, 0, ..., x_n = 1, 0, ...)$ - ненулевое решение $\Rightarrow \sigma_p = \{a_n\}$

$\lambda \notin \sigma_p(A) \Rightarrow \exists (A-\lambda I)^{-1} : (A-\lambda I)^{-1}(y) = (\frac{y_1}{a_1-\lambda}, \frac{y_2}{a_2-\lambda}, ..., \frac{y_n}{a_n-\lambda}, ...)$

если $dom(A-\lambda I)^{-1}=l_2$, то $\lambda \in \rho(A) = \Bbb{C} \setminus \sigma(A)$

(а) $ \Rightarrow dom(A-\lambda I)^{-1}=l_2$ при $ \sup_n |\frac{1}{a_n-\lambda}|<C \Rightarrow \forall n : |a_n - \lambda| \ge \delta \Rightarrow \lambda$ не является предельной точкой $(a_n) \Rightarrow \sigma \subset cl\{a_n\}$

Т.: спектр - замкнутое множество $ \Rightarrow cl\{a_n\} \subset \sigma$

$\sigma \subset cl\{a_n\} \subset \sigma \Rightarrow \sigma = cl\{a_n\}$

$\forall \lambda \in Lim\{a_n\} : dom(A-\lambda I)^{-1} \ne l_2$ т.к. $\lambda \in \sigma$

$(A-\lambda I)^{-1}(y) = (\frac{y_1}{a_1-\lambda}, \frac{y_2}{a_2-\lambda}, ..., \frac{y_n}{a_n-\lambda}, ...)$

$dom(A-\lambda I)^{-1}=\{y \in l_2|(A-\lambda I)^{-1}y \in l_2\}$

$L_n = \{y \in l_2 | (y_1, ..., y_n, 0 , ...)\}$

$\forall n : L_n \subset dom(A-\lambda I)^{-1} \Rightarrow \sum _n=1 ^\infty L_n \subset dom(A- \lambda I)^{-1}$

$\forall x \in l_2 : x=(x_1, x_2, ..., x_n, ...)$ рассмотрим $X_n = (x_1, x_2, ..., x_n, 0, 0, ...) \in L_n$

$||x-X_n|| = \sqrt{\sum_{k=1}^\infty |x_{n+k}|^2}\rightarrow_{n\rightarrow \infty} 0$ т.к. это хвост сходящегося ряда $\sum|x_n|^2 \Rightarrow cl(\sum L_n)=l_2 \Rightarrow cl(dom(A-\lambda I)^{-1})=l_2 \Rightarrow \sigma_c = Lim \{a_n\} \Rightarrow \sigma_r = \varnothing$

б) $(\Rightarrow) $

Т.: если $A$ компактен, то $\forall \epsilon > 0 \exists $ лишь конечное число его собственных значений таких, что $|\lambda| \ge \epsilon$

предположим $a_n \to\!\!\!\!\!\!/\ \ 0$ при $n \to \infty \Leftrightarrow \exists \epsilon > 0 \forall n_0 \exists n \ge n_0, a_n \ge \epsilon$ т.е. $\exists $ бесконечно много $n$ таких, что $|a_n| \ge \epsilon$, а это противоречит теореме

г) $(Ax, y) = (x, By)$

$(Ax, y) = (a_1 x_1 \overline{y_1} + a_2 x_2 \overline{y_2} + ... + a_n x_n \overline{y_n} + ...) = (x_1 \overline{\overline{a_1}y_1} + x_2 \overline{\overline{a_2}y_2} + ... + x_n \overline{\overline{a_n}y_n} + ...) = (x, By)$

где $By = (\overline{a_1}y_1, \overline{a_2}y_2, ..., \overline{a_n}y_n, ...)$

д) $A^* = A \Leftrightarrow \forall n : a_n = \overline{a_n} \Leftrightarrow \forall n : a_n \in \Bbb R$

е) $A$ унитарен, если $(Ax, Ay) = (x, y) \forall x,y \in l_2$

$(Ax, Ay) = ((a_1 x_1, a_2 x_2, ...), (a_1 y_1, a_2 y_2, ...)) = a_1 x_1 \overline{a_1} \overline{y_1} + a_2 x_2 \overline{a_2} \overline{y_2} + ... $

$\forall n : a_n \overline{a_n} = 1$


\end{document}

