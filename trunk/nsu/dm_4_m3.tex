\documentclass[russian]{article}
\usepackage[T1]{fontenc}
\usepackage[utf8]{inputenc}
\usepackage{amsmath,amssymb,esint,babel,qtree,ulem}
\usepackage{graphicx}
\makeatletter
\makeatother
\pagestyle{myheadings}
\markright{Александров А., 3372}

\begin{document}

\section*{3.4}

$\{aa,ab,cc,cca,bcca\}$

\paragraph{а}

\Tree [.$ccabccabccabcc$ [.$cc$ [.$ab$ [.$cc$ [.$ab$ [.$cc$ [.$ab$ [.$cc$  ] ] ] [.$cca$ \sout{$bcc$} ] ] ] [.$cca$ [.$bcca$ \sout{$bcc$} ] ] ] ] [.$cca$ [.$bcca$ [.$bcca$ \sout{$bcc$} ] ] ] ]

\paragraph{б}

\Tree [.$bccaccabccabccacabcca$ [.$bcca$ [.$cc$ [.$ab$ [.$cc$ [.$ab$ [.$cc$ \sout{$acabcca$} ] [.$cca$ \sout{$cabcca$} ] ] ] [.$cca$ [.$bcca$ \sout{$cabcca$} ] ] ] ] [.$cca$ [.$bcca$ [.$bcca$ \sout{$cabcca$} ] ] ] ] ]

\paragraph{в}

\Tree [.$abbccaccabccaabab$ [.$ab$ [.$bcca$ [.$cc$ [.$ab$ [.$cc$ [.$aa$ \sout{$bab$} ] ] [.$cca$ [.$ab$ [.$ab$  ] ] ] ] ] [.$cca$ [.$bcca$ [.$ab$ [.$ab$  ] ] ] ] ] ] ]

\section*{3.9}

пусть $C_1 \cup C_2 / \{e\}$ - зависимо. тогда в множестве $C_1 \cup C_2$ есть независимое $B$ такое, что $|B|=|C_1 \cup C_2| - 1$. если $A=C_1 \cap C_2$, то $A$ - часть цикла, следовательно $A$ - независимое, $|A| < |B|$. по 2 аксиоме можно дополнить его, т.к. максимальные по включению множества имеют одну размерность. тогда получается независимое множество, которое содержит либо цикл $C_1$, либо цикл $C_2$, а значит оно зависимое - противоречие.

\end{document} 