\documentclass[russian,twocolumn]{article}
\usepackage[left=1cm,right=1cm,top=1cm,bottom=1cm,bindingoffset=0cm]{geometry}
\usepackage[T1]{fontenc}
\usepackage[utf8]{inputenc}
\usepackage{amsmath, amssymb, esint, babel, qtree}
\makeatletter
\makeatother

\begin{document}

Понятие файла и файловой системы. Что такое каталог?
Ответ: 
Файл - это совокупность данных, доступ к которой осуществляется по ее имени. 
Таблица преобразования имен в адреса - директория, каталог. 
Совокупность каталогов и других метаданных, т.е. системных структур данных, отслеживающих размещение файлов на диске и свободное дисковое пространство, называется файловой системой. 
Источник: 
Книжка, с.632. 

Определение задачи реального времени.
Ответ: 
Режим реального времени - режим, при котором пользовательская программа получает управление в течении гарантированного (и, как правило, достаточно небольшого) времени после возникновения того или иного внешнего события. 
Источник: 
Книжка, с 16. 

Алгоритм работы библиотечных функций malloc/free языка C.
Ответ: 
Блоки размером более 4096 байт выделяются стратегией first fit из двусвязного кольцевого списка с использованием циклического просмотра, а освобождаются с помощью метода, который в указанном ранее смысле похож на алгоритм парных меток. Блоки меньшего размера объединяются в очереди с размерами, пропорциональными степеням двойки, как в алгоритме близнецов. Элементы этих очередей называются фрагментами. В отличие от алгоритма близнецов, мы не объединяем при освобождении парные фрагменты. Вместо этого мы разбиваем наш 4-кБ блок на фрагменты одинакового размера. Пока хотя бы один фрагмент блока занят, весь блок считается занятым. 
Описатели блоков хранятся не вместе с самими блоками, а в отдельном динамическом массиве _heapinfo. Описатель заводится не на непрерывную последовательность свободных байтов, а на каждые 4096 байт памяти. 
Источник: 
Книжка, с. 231-232 

Что такое системный и пользовательский режимы процессора?
Ответ: 
(???) Режим супервизора — привилегированный режим работы процессора, как правило используемый для выполнения ядра операционной системы. 
В данном режиме работы процессора доступны привилегированные операции, как то: операции ввода-вывода к периферийным устройствам, изменение параметров защиты памяти, настроек виртуальной памяти, системных параметров и прочих параметров конфигурации. Как правило, в режиме супервизора или вообще не действуют ограничения защиты памяти или же они могут быть произвольным образом изменены, поэтому код, работающий в данном режиме, как правило, имеет полный доступ ко всем системным ресурсам (адресное пространство, регистры конфигурации процессора и т. п.). 
Источник: 
Википедия 

Что такое транзакция?
Ответ: 
Группа операций, которые либо исполняются все вместе, либо не исполняются вовсе. 
Источник: 
Книжка, с 401; лекции 

Что такое семафоры Дейкстры?
Ответ: 
Семафор Дейкстры представляет собой целочисленную переменную, с которой ассоциирована очередь ожидающих нитей. Стараясь пройти через семафор, нить пытается вычесть из значения переменной 1. Если значение переменной >= 1, нить проходит сквозь семафор успешно (семафор открыт). Если переменная равна 0 (семафор закрыт), нить останавливается и ставится в очередь. 
Закрытие семафора соответствует захвату объекта или ресурса, доступ к которому контролируется этим семафором. Если объект захвачен, остальные нити вынуждены ждать его освобождения. Закончив работу с объектом (выйдя из критической секции), нить увеличивает значение семафора на единицу, открывая его. При этом первая из стоявших в очереди нитей активизируется и вычитает из значения семафора единицу, снова закрывая семафор. Если же очередь была пуста, то ничего не происходит, просто семафор остается открытым. Тогда первая нить, подошедшая к семафору, успешно пройдет через него. 
Источник: 
Книжка, с 394 

Что такое мертвая блокировка?
Ответ: 
Цикл взаимного ожидания. Критерий блокировки - образование замкнутого цикла в графе ожидающих друг друга задач. 
Источник: 
Книжка, с 386 

Что такое контекст процесса?
Ответ: 
Полный набор регистров, которые нужно сохранить, чтобы нить не заметила переключения, называется контекстом нити или, в зависимости от принятой в конкретной ОС терминологии, контекстом процесса. К таким регистрам, как минимум, относятся все регистры общего назначения, указатель стека, счетчик команд и слово состояния процессора. Если система использует виртуальную память, то в контекст входят также регистры диспетчера памяти, управляющие трансляцией виртуального адреса. 
Источник: 
Книжка, с 438 

Что такое гармонически взаимодействующие последовательные процессы?
Ответ: 
1. Каждый поток (нить) представляет собой независимый программный модуль, для которого создаётся иллюзия чисто последовательного исполнения. 
2. Нити не имеют разделяемых данных. 
3. Все обмены данными и вообще взаимодействие происходят с использованием специальных примитивов, которые одновременно выполняют и передачу данных, и синхронизацию. 
4. Синхронизация, не сопровождающаяся передачей данных, просто лишена смысла - нити, не имеющие разделяемых структур данных, совершенно независимы и не имеют ни критических точек, ни нереентерабельных модулей. 
Источник: 
Книжка, с 405 

Что такое селектор страницы (сегмента) в сегментных и страничных диспетчерах памяти?
Ответ: 
Часть адреса в таких диспетчерах памяти, задающее выбор страницы/сегмента: складывая его с указателем на таблицу трансляции, получаем адрес дескриптора данной страницы/сегмента. 
Источник: 
Книжка, с 282 

Что такое дескриптор страницы (сегмента) в сегментных и страничных диспетчерах памяти?
Ответ: 
Элемент таблицы трансляции, содержащий «права доступа к странице, признак присутствия этой страницы в памяти и физический адрес страницы/сегмента. Для сегментов там также хранится его длина. 
Источник: 
Книжка, с 283 

Что такое абсолютный и относительный загрузчики?
Ответ: 
Абсолютная загрузка — «мы всегда будем загружать программу с одного и того же адреса». Это возможно, если «система может предоставить каждому процессу своё адресное пространство» или же если «система может исполнять в данный момент только один процесс». 
Относительная загрузка — «мы загружаем программу каждый раз с нового адреса». При загрузке необходимо настроить программу на новые адреса. «При использовании в коде программы абсолютной адресации надо найти адресные поля всех команд, использующих такую адресацию, и пересчитать эти поля с учётом реального адреса загрузки. Если в коде применялись косвенно-регистровые, базовые и базово-индексные режимы адресации, следует найти места, где в регистр загружается значение адреса.» 
Источник: 
Книжка, с 160, 162-163 

Что является элементом таблицы перемещений в относительном (перемещаемом) загрузочном модуле?
Ответ: 
«Ассемблер при каждой ссылке на ассемблерный символ генерирует не только "заготовку" адреса в коде, но и запись в таблице перемещений. Эта запись хранит место ссылки на такой символ в коде или данных. При настройке программы на реальный адрес загрузки нам, таким образом, необходимо пройтись по всем объектам, перечисленным в таблице перемещений, и переместить каждую из ссылок — сформировать из заготовки адрес. В качестве "заготовки" адреса обычно используется смещение адресуемого объекта от начала программы и для перемещения оказывается достаточно добавить к "заготовке" адрес загрузки» 
Источник: 
Книжка, с 165 

Что такое позиционно-независимый код?
Ответ: 
Код, в котором используется только относительная адресация, можно загружать с любого адреса без всякой перенастройки. Такой код называется позиционно-независимым. 
Источник: 
Книжка, с 168 

Что такое реентерабельная программа?
Ответ: 
Программный модуль, внутри которого имеется хотя бы одна критическая секция, для которой не обеспечено взаимное исключение, называется нереентерабельным. Соответственно модуль, в котором таких секций нет, или который сам обеспечивает взаимное исключение для них, называется реентерабельным или реентрантным (reentrant). 
Источник: 
Книжка, с 379-380 

Что такое критическая секция?
Ответ: 
Интервал исполнения кода, в течении которого модификация нарушает целостность разделяемой структуры данных, и, наоборот, интервал, в течение которого алгоритм нити полагается на целостность этой структуры, называется критической секцией. 
Источник: 
Книжка, с 378-379 

Кольца доступа и списки контроля доступа.
Ответ: 
Список контроля доступа ассоциируется с объектом или группой объектов и представляет собой таблицу, строки которой соответствуют учётным записям пользователей, а столбцы - отдельным операциям, которые можно совершить над объектом. Перед выполнением операции система ищет идентификатор пользователя в таблице и проверяет, указана ли выполняемая операция в списке его прав. <...> Разработчики системы безопасности могут (и часто бывают вынуждены) предпринимать достаточно сложные меры для сокращения ACL, предлагая те или иные явные и неявные способы объединения пользователей и защищаемых объектов в группы. 
(???) Кольца защиты — архитектура информационной безопасности и функциональной отказоустойчивости, реализующая аппаратное разделение системного и пользовательского уровней привилегий. Структуру привилегий можно изобразить в виде нескольких концентрических кругов. В этом случае системный режим (режим супервизора или нулевое кольцо, т.н. "кольцо 0"), обеспечивающий максимальный доступ к ресурсам, является внутренним кругом, тогда как режим пользователя с ограниченным доступом — внешним. Традиционно семейство микропроцессоров x86 обеспечивает четыре кольца защиты. 
Источник: 
Книжка, с 772-773; википедия 

Кооперативные многозадачные системы и вытесняющая (preemtive) многозадачность.
Ответ: 
Кооперативная многозадачность — планировщик основан на принципе переключения по инициативе активной нити. 
Вытесняющая многозадачность — общее название для всех методов переключения нитей по инициативе системы. 
Источник: 
Книжка, с 435, 437. 

\end{document}

