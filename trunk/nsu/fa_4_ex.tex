\documentclass[russian,twocolumn]{article}
\usepackage[T1]{fontenc}
\usepackage[utf8]{inputenc}
\usepackage{amsmath, amssymb, esint, babel}
\makeatletter
\makeatother

\begin{document}

\section{Геометрия пространств со скалярным произведением}

\paragraph{Линейные пространства со скалярным произведением}

линейное пространство - $L \ne \varnothing$, такое, что:

I. $\forall x,y \in L$ однозначно определен $z \in L : z = x + y$, причем

1) $x + y = y + x$ [коммутативность]

2) $x + (y + z) = (x + y) + z$ [ассоциативность]

3) $\exists 0 \in L : x + 0 = x \forall x \in L$ [$\exists 0$]
 
4) $\forall x \in L \exists (-x) : x + (-x) = 0$ [$\exists (-x)$]

II. $\forall \alpha \in \Bbb C, x \in L \exists \alpha x \in L$, причем

5) $\alpha (\beta x) = (\alpha \beta) x$

6) $1 x = x$

7) $(\alpha + \beta) x = \alpha x + \beta x$

8) $\alpha (x + y) = \alpha x + \alpha y$

скалярное произведение - $f: L^2 \to \Bbb C (\forall x,y \in L)$:

1) $\forall x_{1},x_{2},y\in L,\alpha_{1},\alpha_{2}:(\alpha_{1}x_{1}+\alpha_{2}x_{2},y)=\alpha_{1}(x_{1},y)+\alpha_{2}(x_{2},y)$

2) $\forall x,y\in L:(x,y)=\overline{(y,x)}$ (компл. сопр.)

3) $\forall x\in L:(x,x)\ge0$ и $(x,x)=0$ только при $x=0$

\paragraph{Евклидовы и унитарные пространства}

евклидово - $L:\{x \in \Bbb R^n \}$ 

унитарное - $L:\{x \in \Bbb C^n \}$

\paragraph{Нормированные пространства}

норма - функционал $||.||:L \to [0, +\infty)$:

1) $||x|| \ge 0$, $||x|| = 0 \iff x = 0$

2) $||x + y|| \le ||x|| + ||y|| \forall x,y \in L$

3) $||\alpha x|| = |\alpha| ||x|| \forall x \in L, \alpha \in \Bbb C$

$||x||=\sqrt{(x,x)}$ - обладает свойствами нормы

норма порождена $(x,y)\Rightarrow\forall x,y \in H : ||x+y||^2 + ||x-y||^2 = 2||x||^2 + 2||y||^2$



\paragraph{Метрические пространства}

\paragraph{Гильбертовы пространства}

$L$ - $H$, если оно полно относилельно $||x||=\sqrt{(x,x)}$

\paragraph{Неравенство Коши-Буняковского}

$\forall x,y \in L : |(x,y)|^2 \le (x,x)(y,y)$

\paragraph{Проектирование на замкнутое подпространство в гильбертовом пространстве}

\paragraph{Ортогональность векторов} $(x,y)=0$

\paragraph{Процесс ортогонализации Грамма-Шмидта}

если $x_1,x_2,...,x_n$ - счетная система линейно независимых векторов в $L$, то новые последовательности $y_n=x_n-\sum_{k=1}^{n-1}(x_n,z_k)z_k$ и $z_n=\frac{y_{n}}{||y_{n}||}$ обладают свойствами:

1) система $z_{1},z_{2},...,z_{n}$ ортонормирована

2) $\forall n\in N$ линейная оболочка векторов $z_{n}$ совпадает с линейной оболочкой $x_{n}$

\paragraph{Коэффициенты Фурье относительно ортонормированной системы}

$x_1,x_2,...,x_n \in L$ - ортонормированная система

$x \in L$

$\lambda _k = (x, x_k)$ - коэффициенты Фурье

$\sum_{k=1}^{\infty}\lambda _k x_k$ - ряд фурье вектора $x$

\paragraph{Задача о наилучшем приближении, проектирование на конечномерные подпространства}

если $S \subset L$, то $x \in S$ - ближайший к $ y \in S$ вектор, если $\forall z \in S : ||y-x||\le||y-z||$

лемма: $S \subset H$, $S=cl(S)$, $y \in H \Rightarrow \exists ! x \in S$ ближайший к $y$

лемма: если $S \subset L$, $y \in L$, то $x \in S$ - ортогональная проекция $y$ на $S$ $\iff$ $x \in S$ - ближайший к $y$ 

\paragraph{Неравенство Бесселя}

$\sum _{k=1}^\infty \lambda _k \le ||x||^2$

\paragraph{Полные ортогональные системы и ряды Фурье}

орт-норм $x, x_1, ..., x_n$ - пополнение $x_1, ..., x_n$



\paragraph{Гильбертов базис}

- ортонормированная $x_1,x_2,...,x_n \in L$, такая, что $\forall x \in L : x=\sum _{k=1}^\infty \lambda _k x_k$ 

\paragraph{Равенство Парсеваля}

$\lambda _k = (x,x_k)$

$\mu _k = (y,y_k)$

$(x,y) = \sum_{k=1}^\infty \lambda_k \overline{\mu_k}$

\paragraph{Понятие замкнутой ортогональной системы}

$\{x_n \in L\}$ замкнута, если $\forall x \in L : ||x||^2=\sum _{k=1}^\infty \lambda _k$

\paragraph{Ортогональность и полнота тригонометрической системы функций}

\paragraph{Определение и общие свойства ортогональных многочленов (рекуррентность, расположение нулей)}

\paragraph{Основная информация о многочленах Лежандра, Эрмита, Лагерра - производящие функции, рекуррентные соотношения, диффуры, формулы Родрига}

\paragraph{Многочлены Чебышева и их основные свойства}

\section{Операторы в гильбертовых пространствах}

\paragraph{Линейные операторы, их общие свойства, операции над ними}

\paragraph{Норма оператора, ограниченные операторы}

\paragraph{Обратимость операторов}

\paragraph{Резольвента и спектр}

\paragraph{Классификация точек спектра}

\paragraph{Линейные функционалы в гильбертовом пространстве}

\paragraph{Сопряженное пространство}

\paragraph{Теорема Рисса}

\paragraph{Бра- и кет-векторы}

\paragraph{Операторная функция Грина}

\paragraph{Сопряженный оператор и его св-ва}

\paragraph{Ограниченные свмосопряженные (эрмитовы) операторы}

\paragraph{Квадратичная форма оператора}

\paragraph{Теорема о регулярных значениях эрмитова оператора}

\paragraph{Теорема Рэлея о норме эрмитова оператора}

\paragraph{Компактные операторы}

\paragraph{Дискретность спектра}

\paragraph{Теорема о границах спектра}

\paragraph{Теорема Гильберта-Шмидта о собственном базисе (диагонализируемость) компактного эрмитова оператора}

\paragraph{Вариационный метод Куранта отыскания собственных чисел}

\section{Интегральные уравнения}

\paragraph{Интегральные уравнения Фредгольма}

\paragraph{Теорема о сжимающем отображении в полном метрическом пространстве}

\paragraph{Уравнения с малым параметром}

\paragraph{Ряд Неймана}

\paragraph{Метод последовательных приближений}

\paragraph{Уравнения с вырожденным ядром}

\paragraph{Теоремы Фредгольма}

\paragraph{Интегральные уравнения с вырожденными ядрами}

\paragraph{Теорема Гильберта-Шмидта для интегральных операторов}

\paragraph{Разложение решения интегрального уравнения по собственным функциям ядра}

\paragraph{Разложение ядра интегрального оператора по его собственным функциям (билинейная форма)}

\paragraph{Интегральные уравнения Вольтерра - теорема о существовании и единственности решения}

\section{Вариационное исчисление}

\paragraph{Примеры задач классического вариационного исчисления}

\paragraph{Простейшая задача вариационного исчисления}

\paragraph{Необходимые условия экстремума}

\paragraph{Лемма Лагранжа}

\paragraph{Уравнения Эйлера}

\paragraph{Задачи, допускающие понижение порядка в уравнении Эйлера}

\paragraph{Задачи о брахистохроне и и о поверхности вращения минимальной площади}

\paragraph{Задачи с несколькими переменными}

\paragraph{Гармонические функции, как экстремали интеграла Дирихле}

\paragraph{Уравнение Эйлера для задачи с высшими производными}

\paragraph{Задача с подвижными концами, условия трансверсальности}

\paragraph{Изопериметрическая задача, теорема Эйлера}

\paragraph{Условный экстремум}

\paragraph{Правило множителей Лагранжа}

\paragraph{Вывод уравнения колебаний струны}

\end{document}

