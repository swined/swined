\documentclass[russian,twocolumn]{article}
\usepackage[T1]{fontenc}
\usepackage[utf8]{inputenc}
\usepackage{amsmath, amssymb, esint, babel}
\makeatletter
\makeatother

\begin{document}

\section{Геометрия пространств со скалярным произведением}

\paragraph{Линейные пространства со скалярным произведением}

линейное пространство - $L \ne \varnothing$, такое, что:

I. $\forall x,y \in L$ однозначно определен $z \in L : z = x + y$, причем

1) $x + y = y + x$ [коммутативность]

2) $x + (y + z) = (x + y) + z$ [ассоциативность]

3) $\exists 0 \in L : x + 0 = x \forall x \in L$ [$\exists 0$]
 
4) $\forall x \in L \exists (-x) : x + (-x) = 0$ [$\exists (-x)$]

II. $\forall \alpha \in \Bbb C, x \in L \exists \alpha x \in L$, причем

5) $\alpha (\beta x) = (\alpha \beta) x$

6) $1 x = x$

7) $(\alpha + \beta) x = \alpha x + \beta x$

8) $\alpha (x + y) = \alpha x + \alpha y$

скалярное произведение - $f: L^2 \to \Bbb C (\forall x,y \in L)$:

1) $\forall x_{1},x_{2},y\in L,\alpha_{1},\alpha_{2}:(\alpha_{1}x_{1}+\alpha_{2}x_{2},y)=\alpha_{1}(x_{1},y)+\alpha_{2}(x_{2},y)$

2) $\forall x,y\in L:(x,y)=\overline{(y,x)}$ (компл. сопр.)

3) $\forall x\in L:(x,x)\ge0$ и $(x,x)=0$ только при $x=0$

\paragraph{Евклидовы и унитарные пространства}

евклидово - $L:\{x \in \Bbb R^n \}$ 

унитарное - $L:\{x \in \Bbb C^n \}$

\paragraph{Нормированные пространства}

норма - функционал $||.||:L \to [0, +\infty)$:

1) $||x|| \ge 0$, $||x|| = 0 \iff x = 0$

2) $||x + y|| \le ||x|| + ||y|| \forall x,y \in L$

3) $||\alpha x|| = |\alpha| ||x|| \forall x \in L, \alpha \in \Bbb C$

$||x||=\sqrt{(x,x)}$ - обладает свойствами нормы

норма порождена $(x,y)\Rightarrow\forall x,y \in H : ||x+y||^2 + ||x-y||^2 = 2||x||^2 + 2||y||^2$

\paragraph{Метрические пространства}

множество точек с функцией расстояния $d:M \times M \to \Bbb R$, где $\forall x,y,z \in M$:

$d(x,y) = 0 \iff x=y$

$d(x,y) = d(y,x)$

$d(x,z) \le d(x,y) + d(y,z)$

например: $\Bbb R$, где $d = |x - y|$

\paragraph{Гильбертовы пространства}

$L$ - $H$, если оно полно относилельно $||x||=\sqrt{(x,x)}$

полное $\iff \forall$ фунд. посл. в нем сх.

$x_1,...,x_n \in L$ - фунд., если $\forall \epsilon > 0 \exists n_0 : \forall m,n \ge n_0 : ||x_n - x_m|| < \epsilon $

\paragraph{Неравенство Коши-Буняковского}

$\forall x,y \in L : |(x,y)|^2 \le (x,x)(y,y)$

или $|(x,y)| \le ||x|| ||y||$

\paragraph{Проектирование на замкнутое подпространство в гильбертовом пространстве}

если $S \subset L$, то $x \in S$ - ортогональная проекция $y \in L$ на $S$, если $\forall z \in S : y-x \perp z$ 

если $S \subset L$, то $x \in S$ - ближайший к $ y \in S$ вектор, если $\forall z \in S : ||y-x||\le||y-z||$

лемма: $S \subset H$, $S=cl(S)$, $y \in H \Rightarrow \exists ! x \in S$ ближайший к $y$

лемма: если $S \subset L$, $y \in L$, то $x \in S$ - ортогональная проекция $y$ на $S$ $\iff$ $x \in S$ - ближайший к $y$ 


\paragraph{Ортогональность векторов} $(x,y)=0$

\paragraph{Процесс ортогонализации Грамма-Шмидта}

если $x_1,x_2,...,x_n$ - счетная система линейно независимых векторов в $L$, то новые последовательности $y_n=x_n-\sum_{k=1}^{n-1}(x_n,z_k)z_k$ и $z_n=\frac{y_{n}}{||y_{n}||}$ обладают свойствами:

1) система $z_{1},z_{2},...,z_{n}$ ортонормирована

2) $\forall n\in N$ линейная оболочка векторов $z_{n}$ совпадает с линейной оболочкой $x_{n}$

\paragraph{Коэффициенты Фурье относительно ортонормированной системы}

$x_1,x_2,...,x_n \in L$ - ортонормированная система

$x \in L$

$\lambda _k = (x, x_k)$ - коэффициенты Фурье

$\sum_{k=1}^{\infty}\lambda _k x_k$ - ряд фурье вектора $x$

\paragraph{Задача о наилучшем приближении, проектирование на конечномерные подпространства}

пусть $S \subset L$, тогда $\forall y \in L : x = \sum _{k=1}^n \lambda_{k,y} x_k$ - ортогональная проекция $y$ на $S$, причем $||y||^2 = ||x||^2 + ||y-x||^2$

\paragraph{Неравенство Бесселя}

$\sum _{k=1}^\infty |\lambda _k|^2 \le ||x||^2$

\paragraph{Полные ортогональные системы и ряды Фурье}

$\nexists x \ne 0 \in L : \forall n : x \perp x_n$

\paragraph{Гильбертов базис}

- ортонормированная $x_1,x_2,...,x_n \in L$, такая, что $\forall x \in L : x=\sum _{k=1}^\infty \lambda _k x_k$ 

Т.: во всяком сепарабельном $H \exists $ Г.б., состоящий из конечного или счетного множества векторов

сепарабельное $ \iff \exists $ счетное, плотное подмножество (например $\Bbb R$)

\paragraph{Равенство Парсеваля}

$\lambda _k = (x,x_k)$

$\mu _k = (y,y_k)$

$(x,y) = \sum_{k=1}^\infty \lambda_k \overline{\mu_k}$

\paragraph{Понятие замкнутой ортогональной системы}

$\{x_n \in L\}$ замкнута, если $\forall x \in L : ||x||^2=\sum _{k=1}^\infty |\lambda _k|^2$

\paragraph{Ортогональность и полнота тригонометрической системы функций}

$\frac{1}{\sqrt{2\pi}}, \frac{1}{\sqrt{\pi}}\sin x, \frac{1}{\sqrt{\pi}} \cos x, ..., \frac{1}{\sqrt{\pi}}\sin nx, \frac{1}{\sqrt{\pi}}\cos nx$ 

ортонормирована и полна в $L_2[-\pi, \pi]$ (по равенству Ляпунову)

\paragraph{Ортогональные многочлены}

весовая функция - $h:[a,b]\to \Bbb R : 0 < \int_a^b h(x)ds < +\infty$ 

$L_2^h[a,b]=\{f|\int_a^b f^2(x)h(x) < \infty\}$

$(f,g)_{L_2^h[a,b]}=\int_a^b f(x)g(x)h(x)dx$

ортогонализуем $\{x^n | n \in \Bbb N\}$, получим орт. многочл. с весом $h$ на $[a,b]$

\paragraph{Общие свойства ортогональных многочленов}

1) $\forall Q_n = \sum_0^{n+1} \alpha_k q_k$

2) последовательность орт. мн. опр. весом однозначно

3) $n > m \Rightarrow \int _a^b Q_m(x) q_n(x) h(x) dx = 0$

4) в $L_2^h[-a,a]$ при четной $h$ $\forall n \in \Bbb N,x\in(-a,a):q_n(-x)=(-1)^n q_n(x)$

5) [рекуррентное соотношение] если $q_n = a_n x^n + b_n x^{n-1} + ...$, то $x q_n(x) = \frac{a_n}{a_{n+1}}q_{n+1}(x)+(\frac{b_n}{a_n} - \frac{b_{n+1}}{a_{n+1}}) q_n(x) + \frac{a_{n-1}}{a_n}q_{n-1}(x)$

\paragraph{Расположение нулей}

1) все нули $q_n$ действительны и лежат на $(a,b)$

2) $\forall n : q_n(b) > 0, (-1)^n q_n(a) > 0$

3) $\forall n : q_n, q_{n-1}$ не могут иметь общих корней

4) если $x_0$ - корень $q_n$, то $q_{n-1}(x_0)$ и $q_{n+1}(x_0)$ имеют разные знаки

5) корни $q_n$ и $q_{n+1}$ перемежаются

\paragraph{Основная информация о многочленах Лежандра, Эрмита, Лагерра - производящие функции, рекуррентные соотношения, диффуры, формулы Родрига}

производящая функция - $w(x,t)=\sum _{n=0} ^ {\infty} \frac{q_n(x)}{\alpha_n}t^n$

1) Лежандра: 

$P_n(x)$, $L_2^1[-1,1]$

$w(x,t)=\frac{1}{\sqrt{1 - 2 t x + t^2}} = \sum_{n=0}{\infty}P_n(x)t^n$

$(n+1)P_{n+1}(x) - (2n+1)xP_{n}(x) + n P_{n-1}(x) = 0$

$P'_{n+1}(x) - P'_{n-1}(x)= (2n+1)P_n(x)$

$[(1 -x^2)y']' + n(n+1) y= 0$

$P_n(x) = \frac{1}{2^n n!}\frac{d^n}{dx^n}(x^2 -1 )^n$

2) Эрмита:

$H_n(x)$, $L_2^{e^{-x^2}}[-\infty, +\infty]$

$w(x,t) = e^{2xt-t^2} = \sum_{n=0}{\infty}\frac{H_n(x)}{n!}t^n$

$H_{n+1}(x) - 2xH_n(x) + 2h H_{n-1}(x) = 0$

$H'_n(x)-2nH_{n-1}(x) = 0$

$y'' - 2x y' + 2 n y = 0$

$H_n(x) = (-1)^n e^{x^2} \frac{d^n}{d x^n} e^{-x^2}$

3) Лагерра:

$L_n^\alpha(x)$, $L_2^{x^\alpha e^{-x}}[0,+\infty]$, $\alpha > -1$

$w(x,t) = \frac{1}{(1-t)^{\alpha + 1}}e^{\frac{-xt}{1-t}}=\sum_{n=0}^\infty L_n^\alpha (x) t^n$

$(n+1)L_{n+1}^\alpha (x) + (x-\alpha - 2n -1)L_n^\alpha(x) + (n+\alpha)L_{n-1}^\alpha(x) = 0$

$\frac{d L_n^\alpha}{dx} - \frac{dL_{n-1}^\alpha}{dx} + L_{n-1}^\alpha = 0$

$x y'' + (1 + \alpha - x) y' + n y = 0$

$L_n^\alpha(x) = e^x \frac{x^{-\alpha}}{n!}\frac{d^n}{d x ^n}[e^{-x}x^{n+\alpha}]$

\paragraph{Многочлены Чебышева и их основные свойства}

1) 1 рода: 

$T_n(x)$, $L_2^{\frac{1}{\sqrt{1-x^2}}}[-1,1]$

$w(x,t) = \frac{1-tx}{1 - 2tx + t^2} = \sum_{n=1}^\infty T_n(x)t^n$

$T_0(x) = 1$, $T_1(x) = x$, $T_{n+1}(x) = 2xT_n(x) - T_{n-1}(x)$

$T_n(x) = \cos n \arccos x$

2) 2 рода:

$U_n(x)$, $L_2^{\sqrt{1-x^2}}[-1,1]$

$w(x,t) = \frac{1}{1 - 2tx + t^2} = \sum_{n=1}^\infty U_n(x)t^n$

$U_0 = 1$, $U_1 = 2x$, $U_{n+1}=2xU_n(x)-U_{n-1}(x)$

\section{Операторы в гильбертовых пространствах}

\paragraph{Линейные операторы, их общие свойства, операции над ними}

линейный оператор - $A:H\to H_1:\forall x,y\in H \forall \alpha,\beta \in \Bbb C:A(\alpha x + \beta y) = \alpha A x + \beta A y$

1) $A:H \to H_1, B : H \to H_1 : \forall \alpha, \beta \in \Bbb C : (\alpha A + \beta B)x = \alpha A x + \beta B x$ - линеен

2) $A:H \to H_1, B : H \to H_1 : (BA)x = B(Ax)$ - линеен

\paragraph{Норма оператора, ограниченные операторы}

$||A|| = \sup _{||x|| \le 1} ||Ax|| = \sup _{||x||=1} ||Ax|| = \sup _{||x|| \ne 0} \frac{||Ax||}{||x||}$

Т.: если $A$ линеен, то эквивалентны:

1) $\exists x_0 \in H$, в которой $A$ непрерывен \footnote{$\forall \epsilon > 0 \exists \delta > 0 : \forall x \in H : ||x - x_0|| < \delta : ||Ax - Ax_0|| < \epsilon$}

2) $A$ непрерывен

3) $A$ ограничен \footnote{переводит ограниченное множество в ограниченное. ограниченное множество содержится в шаре конечного радиуса.}

3) $||A|| < \infty$

\paragraph{Обратимость операторов}

$A$ - обратимый $\iff Ax=y$ имеет не более одного решения

образ ($im(A)$) - совокупность $y \in H_1$ таких, что $\exists x \in H$ такой, что $y = Ax$ 

$A^{-1}$ - сопоставляет каждому $y \in im(A)$ единственный $x \in H$ такой, что $Ax=y$

Т. Неймана: если $A:H\to H$ линейный, такой, что $dom(A) = H$ и $||A|| < 1$, то $I-A$ обратим, причем обратный ограничен, определен во всем $H$ и $(I-A)^{-1}=\sum_{n=0}^\infty A^n$, где $A^0 = 0$, $A_{n+1} = AA^n$

\paragraph{Резольвента и спектр}

$y = B^{-1} x = (A-\lambda I) ^ {-1} x$

1) $B$ необратим $\Rightarrow \lambda \in \sigma_p$

2) $B$ обратим, причем $dom(B^{-1}) = H \Rightarrow \lambda \in \rho,B^{-1} = R_\lambda$ 

3) $B$ обратим, но $dom(B^{-1})$ - плотное в $H$ подпространство $\Rightarrow \lambda \in \sigma_c$

4) $B$ обратим, но $dom(B^{-1})$ - не плотное в $H$ подпространство $\Rightarrow \lambda \in \sigma_r$

свойства: 

1) $\sigma = \Bbb C / \rho$

2) $\sigma = \sigma_p \cup \sigma_c \cup \sigma_r$

$\sigma_p \cap \sigma_c = \sigma_p \cap \sigma_r = \sigma_c \cap \sigma_r = \varnothing$

3) $\sigma = cl(\sigma)$

4) $\sigma \subset \{\lambda \in \Bbb C | |\lambda| \le ||A||\}$

\paragraph{Линейные функционалы в гильбертовом пространстве}

$f:H \to \Bbb C$

$ker(f)=\{x\in H | f(x) = 0\}$

свойства:

1) $f$ - лин. ф-л $\Rightarrow ker(f) \subset H$ 

2) $f$ непр $\iff ker(f) = cl(ker(f))$

3) $f \ne 0 \Rightarrow dim(ker^\perp (f)) = 1$

\paragraph{Сопряженное пространство}

$H^*$ - множество всех нлф определенных на $H$

\paragraph{Теорема Рисса}

1) $\forall$ нлф $\exists!x_0\in H : f(x) = (x, x_0)\forall x\in H$, причем $||f||=||x_0||$

2) $\forall x_0 \in H \Rightarrow f(x) = (x,x_0)$ - нлф, $||f||=||x_0||$

\paragraph{Бра- и кет-векторы}

$\langle x|y \rangle = \{\langle x | \} \{ | y \rangle \}$

линейны по второму аргументу

бра-вектор - $\langle x | $ 

кет-вектор - $| y \rangle $

если $x_1,...,x_n,...$ - ортонормированный базис, то $\sum _{n=1}^\infty |x_n \rangle \langle x_n| = I$

\paragraph{Операторная функция Грина}

$A:H \to H$ - линейный оператор

$x_1,...,x_n$ - собственные векторы $A$, образующие ортонормированный базис

$\lambda_j : Ax_j = \lambda_j x_j$

$R_\lambda = \sum_{j=1}^\infty \frac{|x_j \rangle \langle x_j|}{\lambda_j - \lambda}$

\paragraph{Сопряженный оператор и его св-ва}

$\forall x,y \in H : (Ax,y) = (x,By) \iff A^* = B$

0) $A^*$ существует и единственный

1) $A^*$ линейный ограниченный оператор

2) $(\alpha A + \beta B)^* = \overline{\alpha}A^* + \overline{\beta}B^*$

3) $(A^*)^* = A$

4) $||A^*|| = ||A||$

5) $I^* = I$

лемма: $A:H \to H$ - лин. огр. оп. и $\lambda \notin \sigma_p(A)$, тогда $\lambda \in \sigma_r (A) \iff \overline{\lambda} \in \sigma_p(A^*)$ 

\paragraph{Ограниченные самосопряженные (эрмитовы) операторы}

$A^* = A$ - линейный самосоп. оп.

Т.: $\sigma_p(A) \subset \Bbb R$, собст. векторы. орт. м-ду собой

$S \subset H$ - инвариантное подпространство, если $\forall x \in S : Ax \in S$

Т. $S$ - инв. подпр. $\Rightarrow S^\perp $ - инв. подпр. 

\paragraph{Квадратичная форма оператора} $(Ax, x)$

\paragraph{Теорема о регулярных значениях эрмитова оператора}

\paragraph{Теорема Рэлея о норме эрмитова оператора}

Т.: $||A|| = \sup _{||x|| \le 1} |(Ax,x)|$

\paragraph{Компактные операторы}

$A:H \to H_1$ - компактный, если $\exists A_n : H \to H_1 , dim(im(A_n)) < \infty, A_n$ непр. и $A_n \to _{n \to \infty} A$

1) $A,B$ - комп. $\Rightarrow \alpha A + \beta B$ - комп.

2) $A$ - комп. $\Rightarrow A$ - ограничен

3) $dim(H) = \infty \Rightarrow I $ - не комп.

4) $A:H \to H_1$ комп. и $B:H_1 \to H_2, C : H_3 \to H$, то $BA, AC$ - комп.

5) $dim(H) = \infty,A$ комп. и огр., тогда $A^{-1}$ не огр., т.е. $0 \in \sigma(A)$

6) $A_n$ - комп., $A_n \to A \Rightarrow A$ - комп.

\paragraph{Дискретность спектра}
 
$A:H\to H$ - комп. $\Rightarrow \forall \epsilon > 0 \exists $ лишь конечное число лин. нез. собст. век. оп. $A$ отвечающих собст. знач. $|\lambda| \ge \epsilon$

следствия:

1) собственному знач. $\lambda \ne 0$ отвечает конечное число лин. нез. собств. век. (геом. кратность собст числа)

2) $\forall \epsilon > 0 \exists $ лишь конечно число собст. чисел таких, что $|\lambda| \ge \epsilon$

3) собст. знач. комп. оп. можно перенумеровать в порядке невозрастания модулей с учетом кратности

\paragraph{Теорема о границах спектра}

$A$ - комп. самосоп. оп., тогда $\sigma \subset [m,M] \subset \Bbb R$, где $m = \inf _{||x||=1}(Ax.x)$ и $M = \sup _{||x||=1}(Ax.x)$

$m,M$ - границы оператора

\paragraph{Теорема Гильберта-Шмидта о собственном базисе (диагонализируемость) компактного эрмитова оператора}

пусть $dim(H) > 0$, $A: H \to H$ - комп. самосоп. оп., тогда в $H \exists $ ортонорм. базис, сост. из собст. векторов оп. $A$

\paragraph{Вариационный метод Куранта отыскания собственных чисел}

$H$ - сепарабельное гильбертово пространство

$A:H \to H$ - комп. самосоп. лин. оп.

$\forall n \ge 1 : \lambda_n = \inf _{H_{n-1}} \sup _{x\ne 0, x \perp H_{n-1}} \frac{(Ax,x)}{||x||^2}, \lambda_{-n} = \sup _{H_{n-1}} \inf _{x\ne 0, x \perp H_{n-1}} \frac{(Ax,x)}{||x||^2}$

$H_{n-1} \subset H : dim(H_{n-1}) = n -1$

\section{Интегральные уравнения}

\paragraph{Интегральные уравнения Фредгольма}

1 рода: $int_a^b K(t,s)x(s) ds + f(t) = 0$

2 рода: $int_a^b K(t,s)x(s) ds + f(t) = x(s)$

\paragraph{Теорема о сжимающем отображении в полном метрическом пространстве}

сжимающее отображение - $f:M \to M:\forall x,y \in M : d(f(x), f(y)) < d(x, y)$

пусть $(M,d)$ - полное, $f:M \to M$ - сжимающее, тогда $\exists ! x_0 \in M : f(x_0) = x(0)$

\paragraph{Уравнения с малым параметром}

$\mu int_a^b K(t,s)x(s) ds + f(t) = x(s)$

$x= \mu A x + f$

$x = (I - \mu A)^{-1} f$

Т. Неймана: при $|\mu| < \frac{1}{||A||}$ оп. $(I - \mu A)$ обратим и $(I - \mu A)^{-1} = \sum_{n=0}^{\infty} \mu^n A^n$, т.е. $x = \sum_{n=0}^{\infty} \mu^n A^n x$

\paragraph{Ряд Неймана}

$x = \sum_{n=0}^{\infty} \mu^n A^n x = f + \mu A f + \mu ^ 2 A(Af) + ...$

\paragraph{Метод последовательных приближений}

$x_n = \mu^n A^n f$

$x_0 = f$

$x_{n+1} = \mu A x_n$

$x = \sum_{n=0}^m x_n$ - частичная сумма ряда Неймана, приближенное решение

\paragraph{Уравнения с вырожденным ядром}

$K(t,s) = \sum_{i=0}^n P_i(t) Q_i(s)$ - вырожденное ядро ур. Ф. 2 рода

$x(t) = \sum_{i=1}^nP_i(t) \int _a^b Q_i(s)x(s)ds + f(t)$

$q_i = \int _a^b Q_i(s)x(s)ds$

$x(t) = \sum_{i=1}^nP_i(t) q_i + f(t)$

$x(t) = \sum_{i=1}^nP_i(t) \int _a^b Q_i(s)[\sum_{i=1}^nP_i(s) q_i + f(s)]ds + f(t)$

$a_{i,k} = \int _a^b Q_i P_k(s)ds, b_i = \int_a^b Q_i(s)f(s) ds$

$\sum _{i=1}^n q_i P_i(t) = \sum _{i=1}^nP_i(t)[\sum_{k=1}^n a_{i.k} q_k + b_i]$

$q_i = \sum_{k=1}^n a_{i.k} q_k + b_i$

\paragraph{Теоремы Фредгольма}

	пусть $H$ - гильбертово пространство, $A:H \to H$ - линейный компактный оператор. разрешимость $x-Ax=f$ устанавливается с помощью однородного уравнения $x-Ax=0$ и однородного сопряженного уравнения $y-A^*y=0$:
	
	для уравнения (н) возможны два случая:
	
	1. если (со) имеет только нулевое решение, то (н) имеет единственное решение для любого $f$
	
	2. если (со) имеет $n$ линейно-независимых решений $\{x_n\}$ и (о) имеет ровно $n$ линейно-незавимых решений $\{y_n\}$, то (н) разрешимо если и только если $(y_k,f) = 0 \forall k\in[1,n]$	

\paragraph{Теорема Гильберта-Шмидта для интегральных операторов}

если $f\in L_2[a,b]$ представима через симм. ядро $K \in L_2([a,b]x[a,b])$, то она может быть разложена в ряд $f(t) = \sum_n f_n x_n (t)$, где $\{x_n\}$ ортонорм. посл. собст. ф-ций ур. Ф., а $f_n = \int _a^bf(t)\overline{x_n(t)}dt$

\paragraph{Разложение решения интегрального уравнения по собственным функциям ядра}

$x_n = \mu_n A x_n$

$x(t) = f(t) + \mu \sum _n \frac{x_n(t)}{\mu_n - \mu} \int _a^b \overline{x_n(s)}f(s) ds$

\paragraph{Разложение ядра интегрального оператора по его собственным функциям (билинейная форма)}

для симметричного ядра в $L_2([a,b]x[a,b])$ имеет место $K_n(t,s) = \sum_i\frac{x_i(t)\overline{x_j(t)}}{\mu_i^n}$

\paragraph{Интегральные уравнения Вольтерра}

1 рода: $int_a^t K(t,s)x(s) ds + f(t) = 0$

2 рода: $int_a^t K(t,s)x(s) ds + f(t) = x(s)$

\paragraph{Теорема о существовании и единственности решения}

если $K \in L_2([a,b]x[a,b])$, то $\forall \mu \forall f \in L_2[a,b]$ ур. В. 2 рода $\exists!x\in L_2[a,b]$ - решение, которое можно найти методом посл. приближений

\section{Вариационное исчисление}

\paragraph{Примеры задач классического вариационного исчисления}

задача о брахистохроне:

минимизировать $I[y]=\int_{x_0}^{x_1}\frac{\sqrt{1+(y'(x))^2}}{\sqrt{2g(y_0-y(x))}}dx$

$\begin{cases}
x=\frac{c_1}{2}(t-\sin t) + C_2 \\
y = y_0 - \frac{c_1}{2}(1 - \cos t)
\end{cases}$

задача о поверхности вращения наименьшей площади:

минимизировать $I[y]=\int_{x_0}^{x_1}y(x)\sqrt{1+(y'(x))^2}dx$

$y(x) = C_1 \ch \frac{x-C_2}{C_1}$

\paragraph{Простейшая задача вариационного исчисления}

$I[y]=\int_{x_0}^{x_1}F(x,y(x),y'(x))dx$

\paragraph{Лемма Лагранжа}

если $f:(x_0,x_1) \to \Bbb R$ непр. и $\int_{x_0}^{x_1}f(x)\eta(x)dx=0 \forall$ финитной $\eta : [x_0, x_1] \to \Bbb R$, то $f$ - тождественный нуль

\paragraph{Уравнения Эйлера}

	если $y$ - экстремаль функционала $L(y)=\int_a^b F(x,y(x),y'(x))dx$ при $y(a)=y_0$, $y(b)=y_1$, то $y$ - решение уравнения Эйлера $\frac{\partial F}{\partial y} - \frac{d}{dx}\frac{\partial F}{\partial y'} = 0$

\paragraph{Задачи, допускающие понижение порядка в уравнении Эйлера}

1) $F=F(x,y)$, тогда $F_y(x,y) = 0$

2) $F=F(x,y')$, тогда $\frac{dF_{y'}}{dx} = 0$

3) $F=F(y,y')$, тогда $y'F_{y'}-F=C$

$\frac{d}{dx}[y'F_{y'}-F]=y''F_{y'}+y'\frac{d}{dx}F_{y'}-F_{y}y'-F_{y'}y''=-y'[F_y-\frac{d}{dx}F_{y'}]$

\paragraph{Задачи с несколькими переменными}

записываем ур. Э. для каждой переменной по отдельности

\paragraph{Гармонические функции, как экстремали интеграла Дирихле}

$I[z]=\iint_D F(x,y,z(x,y),z_x(x,y),z_y(x,y)) dx dy$

$F_z - \frac{\partial}{\partial x}F_{z_x}-\frac{\partial}{\partial y}F_{z_y} = 0$

$I[z]=\iint_D [(\frac{\partial z}{\partial x})^2 + (\frac{\partial z}{\partial y})^2] dx dy$

$z_{xx} + z_{yy} = 0$

\paragraph{Уравнение Эйлера для задачи с высшими производными}

$F_y + (-1)^n \frac{d^n}{dx^n}F_{y^{(n)}}=0$

\paragraph{Задача с подвижными концами, условия трансверсальности}

найти экстремум функционала при $y(x_0) = y_0$ и $y(x_1) = \varphi(x_1)$

решаем $F_y - \frac{d}{dx}F_{y'}=0$

$\begin{cases}
y(x_0, C_1, C_2) = y_0 \\
y(x_0, C_1, C_2) = \varphi(x_0) \\
F(x_1)-F_{y'}(x_1)[y'(x_1)-\varphi'(x_1)] = 0
\end{cases}$

\paragraph{Изопериметрическая задача, теорема Эйлера}

$J[y]=\int_{x_0}^{x^1}G(x,y(x),y'(x))dx=const$

минимизировать $I[y]=\int_{x_0}^{x^1}F(x,y(x),y'(x))dx$
 
если $y(x)$ является решением $J[y]=\int_{x_0}^{x_1}G(x, y, y')dx = const$ и не является экстремалью функционала $I[y] = \int_{x_0}^{x_1}F(x, y, y')dx$, то $\exists \lambda \in \Bbb R$ такое, что $y$ является экстремалью $\widetilde{I}[y]=\int_{x_0}^{x^1}(F-\lambda G) dx$

\paragraph{Условный экстремум}

минимизировать $I[y,z] = \int_{x_0}^{x_1}F(x,y(x),y'(x),z(x),z'(x))dx$ при $G(x,y(x),z(x)) = 0$

\paragraph{Правило множителей Лагранжа}

если функции $y=y(x)$ и $z=z(x)$ являются решением простейшей вариационной задачи на условный экстремум, то $\exists$ непр. диф. ф. $\lambda:(x_0,x_1)\to \Bbb R$ такая, что $y$ и $z$ являются экстремалями $I^*[y,z,\lambda]=\int_{x_0}^{x_1}F^*(x,y,y',z,z',\lambda)dx$, где $F^*=F+\lambda(x)G$

\paragraph{Вывод уравнения колебаний струны}

$y(x,t)$ - отклонение

$ \frac{\partial ^2 y}{\partial t^2} = a^2\frac{\partial ^2 y}{\partial x^2}$

\end{document}

