\documentclass[russian,twocolumn]{article}
\usepackage[T1]{fontenc}
\usepackage[utf8]{inputenc}
\usepackage{amsmath, amssymb, esint, babel}
\makeatletter
\makeatother

\begin{document}

\section{Геометрия пространств со скалярным произведением}

\paragraph{Линейные пространства со скалярным произведением}

\paragraph{Евклидовы и унитарные пространства}

\paragraph{Нормированные пространства}

\paragraph{Метрические пространства}

\paragraph{Гильбертовы пространства}

\paragraph{Неравенство Коши-Буняковского}

\paragraph{Проектирование на замкнутое подпространство в гильбертовом пространстве}

\paragraph{Ортогональность векторов}

\paragraph{Процесс ортогонализации Грамма-Шмидта}

\paragraph{Коэффициенты Фурье относительно ортонормированной системы}

\paragraph{Задача о наилучшем приближении, проектирование на конечномерные подпространства}

\paragraph{Неравенство Бесселя}

\paragraph{Полные ортогональные системы и ряды Фурье}

\paragraph{Гильбертовы базисы}

\paragraph{Равенство Парсеваля}

\paragraph{Понятие замкнутой ортогональной системы}

\paragraph{Ортогональность и полнота тригонометрической системы функций}

\paragraph{Определение и общие свойства ортогональных многочленов (рекуррентность, расположение нулей)}

\paragraph{Основная информация о многочленах Лежандра, Эрмита, Лагерра - производящие функции, рекуррентные соотношения, диффуры, формулы Родрига}

\paragraph{Многочлены Чебышева и их основные свойства}

\section{Операторы в гильбертовых пространствах}

\paragraph{Линейные операторы, их общие свойства, операции над ними}

\paragraph{Норма оператора, ограниченные операторы}

\paragraph{Обратимость операторов}

\paragraph{Резольвента и спектр}

\paragraph{Классификация точек спектра}

\paragraph{Линейные функционалы в гильбертовом пространстве}

\paragraph{Сопряженное пространство}

\paragraph{Теорема Рисса}

\paragraph{Бра- и кет-векторы}

\paragraph{Операторная функция Грина}

\paragraph{Сопряженный оператор и его св-ва}

\paragraph{Ограниченные свмосопряженные (эрмитовы) операторы}

\paragraph{Квадратичная форма оператора}

\paragraph{Теорема о регулярных значениях эрмитова оператора}

\paragraph{Теорема Рэлея о норме эрмитова оператора}

\paragraph{Компактные операторы}

\paragraph{Дискретность спектра}

\paragraph{Теорема о границах спектра}

\paragraph{Теорема Гильберта-Шмидта о собственном базисе (диагонализируемость) компактного эрмитова оператора}

\paragraph{Вариационный метод Куранта отыскания собственных чисел}

\section{Интегральные уравнения}

\paragraph{Интегральные уравнения Фредгольма}

\paragraph{Теорема о сжимающем отображении в полном метрическом пространстве}

\paragraph{Уравнения с малым параметром}

\paragraph{Ряд Неймана}

\paragraph{Метод последовательных приближений}

\paragraph{Уравнения с вырожденным ядром}

\paragraph{Теоремы Фредгольма}

\paragraph{Интегральные уравнения с вырожденными ядрами}

\paragraph{Теорема Гильберта-Шмидта для интегральных операторов}

\paragraph{Разложение решения интегрального уравнения по собственным функциям ядра}

\paragraph{Разложение ядра интегрального оператора по его собственным функциям (билинейная форма)}

\paragraph{Интегральные уравнения Вольтерра - теорема о существовании и единственности решения}

\section{Вариационное исчисление}

\paragraph{Примеры задач классического вариационного исчисления}

\paragraph{Простейшая задача вариационного исчисления}

\paragraph{Необходимые условия экстремума}

\paragraph{Лемма Лагранжа}

\paragraph{Уравнения Эйлера}

\paragraph{Задачи, допускающие понижение порядка в уравнении Эйлера}

\paragraph{Задачи о брахистохроне и и о поверхности вращения минимальной площади}

\paragraph{Задачи с несколькими переменными}

\paragraph{Гармонические функции, как экстремали интеграла Дирихле}

\paragraph{Уравнение Эйлера для задачи с высшими производными}

\paragraph{Задача с подвижными концами, условия трансверсальности}

\paragraph{Изопериметрическая задача, теорема Эйлера}

\paragraph{Условный экстремум}

\paragraph{Правило множителей Лагранжа}

\paragraph{Вывод уравнения колебаний струны}

\end{document}

