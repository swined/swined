\documentclass[russian]{article}
\usepackage[T1]{fontenc}
\usepackage[utf8]{inputenc}
\usepackage{amssymb,esint,babel,qtree,ulem}
\makeatletter
\makeatother
\pagestyle{myheadings}
\markright{Александров А., 3372}

\begin{document}

\section*{2.1}

\paragraph*{а}

ищем кратчайшее ребро. если оно не образует цикл с выбранными ранее, добавляем его в дерево, иначе выбрасываем из таблицы.

\begin{tabular}{|c|c|c|c|c|c|c|c|}\hline
    & 1  & 2  & 3  & 4	& 5	 & 6  & 7\\\hline
1	& -- & {\bf 36} & 44 & {\bf 37} & 41 & 53 & 43\\\hline
2	&    & -- & 48 & 57 & 43 & 45 & 51\\\hline
3	&    &    & -- & 39 & 39 & {\bf 42} & {\bf 38}\\\hline
4	&    &    &    & -- & 52 & 51 & {\bf 34}\\\hline
5	&    &    &    &    & -- & 49 & {\bf 35}\\\hline
6	&    &    &    &    &    & -- & 45\\\hline
7	&    &    &    &    &    &    & -- \\\hline
\end{tabular}

\Tree [.4 7 ]

\Tree [.7 4 5 ]

\Tree [.7 4 5 ]
\Tree [.1 2 ]

\Tree [.4 [.7 5 ] [.1 2 ] ]

\Tree [.4 [.7 3 5 ] [.1 2 ] ]

выкидываем ребра (3,4), (3,5), (4,5), т.к. они образуют циклы

\Tree [.4 [.7 [.3 6 ] 5 ] [.1 2 ] ]

\paragraph*{б}

пусть полученное дерево $T_s$ неоптимально, а значит существует оптимальное дерево $T_o$. количество ребер $T_s$ и $T_o$ совпадает.
$E(T_s) = \{e_1, ..., e_{n-1}\}$, $w_i = w(e_i)$. $E(T_o) = \{e_1^*, ..., e_{n-1}^*\}$, $w_i* = w(e_i^*)$.
$w(T_s)=\sum_{i=1}^{n-1}w_i$, $w_i \ge w_j \forall i>j$. $w(T_o)=\sum_{i=1}^{n-1}w_i^*$, $w_i^* \ge w_j^* \forall i>j$.
$w(T_s) > w(T_o) \Rightarrow \exists min k : w_k > w_k^*$, но это невозможно, т.к. $w_k > w_{k-1}^* \ge ... $ .
в момент $k$ работы алгоритма найдется $k$ ребер, образующих лес и более легких, чем $e_k$. 
по лемме о пополнении леса в ${e_1^*, ..., e_k^*}$ существует ребро, не входящее в $\{e_1, ..., e_k\}$ и такое, что добавление его в $\{e_1, ..., e_k\}$ порождает лес $\Rightarrow$ противоречие.

\section*{2.2}

\paragraph*{а} алгоритм Дейкстры:

($\infty$, 0, $\infty$, $\infty$, $\infty$, $\infty$, $\infty$)

(56, \sout{0}, $\infty$, 54, $\infty$, $\infty$, 34)

(56, \sout{0}, $\infty$, 37, $\infty$, 77, \sout{34})

(56, \sout{0}, 42, \sout{37}, 76, 67, \sout{34})

(56, \sout{0}, \sout{42}, \sout{37}, 76, 67, \sout{34})

(\sout{56}, \sout{0}, \sout{42}, \sout{37}, 76, 67, \sout{34})

(\sout{56}, \sout{0}, \sout{42}, \sout{37}, 76, \sout{67}, \sout{34})

(\sout{56}, \sout{0}, \sout{42}, \sout{37}, \sout{76}, \sout{67}, \sout{34})

(56, 0, 42, 37, 76, 67, 34)

\paragraph*{б} доказательство корректности алгоритма Дейкстры:

пусть $l(v)$ — длина кратчайшего пути из вершины $a$ в вершину $v$. докажем по индукции, что в момент посещения любой вершины $z$, $d(z)=l(z)$.

база: первой посещается вершина $a$. в этот момент $d(a)=l(a)=0$.

шаг: пускай мы выбрали для посещения вершину $z \ne a$. докажем, что в этот момент $d(z)=l(z)$. для начала отметим, что для любой вершины $v$, всегда выполняется $d(v) \ge l(v)$ (алгоритм не может найти путь короче, чем кратчайший из всех существующих). пусть $P$ — кратчайший путь из $a$ в $z$, $y$ — первая непосещённая вершина на $P$, $x$ — предшествующая ей (следовательно, посещённая). поскольку путь $P$ кратчайший, его часть, ведущая из $a$ через $x$ в $y$, тоже кратчайшая, следовательно $l(y)=l(x)+w(xy)$. по предположению индукции, в момент посещения вершины $x$ выполнялось $d(x)=l(x)$, следовательно, вершина $y$ тогда получила метку не больше чем $d(x)+w(xy)=l(x)+w(xy)=l(y)$. следовательно, $d(y)=l(y)$. с другой стороны, поскольку сейчас мы выбрали вершину $z$, её метка минимальна среди непосещённых, то есть $d(z) \le d(y) = l(y) \le l(z)$. комбинируя это с $d(z) \ge l(z)$, имеем $d(z)=l(z)$, что и требовалось доказать.

поскольку алгоритм заканчивает работу, когда все вершины посещены, в этот момент $d=l$ для всех вершин.

\paragraph*{в} алгоритм Флойда-Уоршелла:

\begin{tabular}{|c|c|c|c|c|c|c|c|}\hline
& 1& 2& 3& 4& 5& 6& 7\\\hline
1&0&$\infty$&23&$\infty$&$\infty$&66&$\infty$\\\hline
2&56&0&$\infty$&54&$\infty$&$\infty$&34\\\hline
3&$\infty$&56&0&$\infty$&$\infty$&$\infty$&$\infty$\\\hline
4&$\infty$&$\infty$&5&0&39&30&$\infty$\\\hline
5&34&$\infty$&$\infty$&$\infty$&0&$\infty$&46\\\hline
6&$\infty$&$\infty$&73&31&19&0&4\\\hline
7&$\infty$&$\infty$&$\infty$&3&$\infty$&43&0\\\hline
\end{tabular}

\begin{tabular}{|c|c|c|c|c|c|c|c|}\hline
& 1& 2& 3& 4& 5& 6& 7\\\hline
1&0&$\infty$&23&$\infty$&$\infty$&66&$\infty$\\\hline
2&56&0&79&54&$\infty$&122&34\\\hline
3&$\infty$&56&0&$\infty$&$\infty$&$\infty$&$\infty$\\\hline
4&$\infty$&$\infty$&5&0&39&30&$\infty$\\\hline
5&34&$\infty$&57&$\infty$&0&100&46\\\hline
6&$\infty$&$\infty$&73&31&19&0&4\\\hline
7&$\infty$&$\infty$&$\infty$&3&$\infty$&43&0\\\hline
\end{tabular}

\begin{tabular}{|c|c|c|c|c|c|c|c|}\hline
& 1& 2& 3& 4& 5& 6& 7\\\hline
1&0&$\infty$&23&$\infty$&$\infty$&66&$\infty$\\\hline
2&56&0&79&54&$\infty$&122&34\\\hline
3&112&56&0&110&$\infty$&178&90\\\hline
4&$\infty$&$\infty$&5&0&39&30&$\infty$\\\hline
5&34&$\infty$&57&$\infty$&0&100&46\\\hline
6&$\infty$&$\infty$&73&31&19&0&4\\\hline
7&$\infty$&$\infty$&$\infty$&3&$\infty$&43&0\\\hline
\end{tabular}

\begin{tabular}{|c|c|c|c|c|c|c|c|}\hline
& 1& 2& 3& 4& 5& 6& 7\\\hline
1&0&79&23&133&$\infty$&66&113\\\hline
2&56&0&79&54&$\infty$&122&34\\\hline
3&112&56&0&110&$\infty$&178&90\\\hline
4&117&61&5&0&39&30&95\\\hline
5&34&113&57&167&0&100&46\\\hline
6&185&129&73&31&19&0&4\\\hline
7&$\infty$&$\infty$&$\infty$&3&$\infty$&43&0\\\hline
\end{tabular}

\begin{tabular}{|c|c|c|c|c|c|c|c|}\hline
& 1& 2& 3& 4& 5& 6& 7\\\hline
1&0&79&23&133&172&66&113\\\hline
2&56&0&59&54&93&84&34\\\hline
3&112&56&0&110&149&140&90\\\hline
4&117&61&5&0&39&30&95\\\hline
5&34&113&57&167&0&100&46\\\hline
6&148&92&36&31&19&0&4\\\hline
7&120&64&8&3&42&33&0\\\hline
\end{tabular}

\begin{tabular}{|c|c|c|c|c|c|c|c|}\hline
& 1& 2& 3& 4& 5& 6& 7\\\hline
1&0&79&23&133&172&66&113\\\hline
2&56&0&59&54&93&84&34\\\hline
3&112&56&0&110&149&140&90\\\hline
4&73&61&5&0&39&30&85\\\hline
5&34&113&57&167&0&100&46\\\hline
6&53&92&36&31&19&0&4\\\hline
7&76&64&8&3&42&33&0\\\hline
\end{tabular}

\begin{tabular}{|c|c|c|c|c|c|c|c|}\hline
& 1& 2& 3& 4& 5& 6& 7\\\hline
1&0&79&23&97&85&66&70\\\hline
2&56&0&59&54&93&84&34\\\hline
3&112&56&0&110&149&140&90\\\hline
4&73&61&5&0&39&30&34\\\hline
5&34&113&57&131&0&100&46\\\hline
6&53&92&36&31&19&0&4\\\hline
7&76&64&8&3&42&33&0\\\hline
\end{tabular}

\begin{tabular}{|c|c|c|c|c|c|c|c|}\hline
& 1& 2& 3& 4& 5& 6& 7\\\hline
1&0&79&23&73&85&66&70\\\hline
2&56&0&42&37&76&67&34\\\hline
3&112&56&0&93&132&123&90\\\hline
4&73&61&5&0&39&30&34\\\hline
5&34&110&54&49&0&79&46\\\hline
6&53&68&12&7&19&0&4\\\hline
7&76&64&8&3&42&33&0\\\hline
\end{tabular}

\paragraph*{г} доказательство корректности:

по индукции

база: $d_0(i,j)=w_{i,j}$

шаг: если кратчайшее расстояние от $i$ до $j$ через $\{1, ..., k+1\}$ не проходит через $k+1$, то верно

иначе $min(d_k(i, k+1), d_k(k+1, j)) \le d_k(i, j) \Rightarrow d_{k+1}(i,j)$ - искомое для этого шага



\end{document} 