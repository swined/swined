\documentclass[russian,twocolumn]{article}
\usepackage[left=1cm,right=1cm,top=1cm,bottom=1cm,bindingoffset=0cm]{geometry}
\usepackage[T1]{fontenc}
\usepackage[utf8]{inputenc}
\usepackage{amsmath, amssymb, esint, babel, qtree}
\makeatletter
\makeatother

\begin{document}

\section{Алгоритмы и их сложность}

\paragraph{Понятие о сложности алгоритма}

\paragraph{Жадные алгоритмы для задачи раскраски и построения гамильтонова пути}

\paragraph{Поиск по графу}

\paragraph{Алгоритмы быстрой сортировки}

\paragraph{AVL-деревья}

добавление: вставляем как в бинарное, потом балансируем всех родителей до корня

простой правый поворот ($b(A) = 2, b(B) \ne -1$):

\Tree [.A L [.B C R ] ]
\Tree [.B [.A L C ] R ]

двойной правый поворот ($b(A) = 2, b(B) = -1$):

\Tree [.A L [.B [.C M N ] R ] ]
\Tree [.C [.A L M ] [.B N R ] ]

простой левый поворот ($b(A) = -2, b(B) \ne 1$):

\Tree [.A [.B L C ] R ]
\Tree [.B L [.A C R ] ]

двойной левый поворот ($b(A) = -2, b(B) = 1$):

\Tree [.A [.B L [.C M N ] ] R ]
\Tree [.C [.B L M ] [.A N R ] ]

удаление: если вершина лист, удаляем её и делаем балансировку всех родителей до корня. если нет, найдём самую близкую по значению вершину в поддереве наибольшей высоты и удалим её, поставив на место удаляемой.

\paragraph{Идея динамического программирования на примере распределительной и обратной к ней задач}

\paragraph{Задача о кратчайшем пути}

\paragraph{Алгоритмы Дейкстры и Флойда-Уоршелла}

\paragraph{Метод ветвей и границ на примере задачи коммивояжера}

\paragraph{Алгоритм Краскалла для задачи о кратчайшей связывающей сети}

\section{Потоки в сетях, частично упорядоченные множества и линейное программирование}

\paragraph{Сети и потоки в них}

\paragraph{Теорема о максимальном потоке и минимальном разрезе}

\paragraph{Алгоритмы для нахождения максимального потока}

\paragraph{Использование сетевых моделей для нахождения связности графов}

\paragraph{Теоремы Менгера и Уитни}

\paragraph{Задача о наибольшем паросочетании в двудольном графе как задача о максимальном потоке в сети}

\paragraph{Разбиение частично упорядоченных множеств на цепи и антицепи}

\paragraph{Обобщение на ориентированные графы}

\paragraph{Теоремы Галлаи-Мильграма, Руа и Дилворта}

\paragraph{Задача о максимальном потоке в сети как задача линейного программирования}

\paragraph{Симплекс-метод решения задачи линейного програмирования}

\section{Элементы теории кодирования}

\paragraph{Постановка задач теории кодирования}

\paragraph{Алгоритмы распознавания однозначности декодирования алфавитных кодов}

\paragraph{Неравенство Крафта-Макмиллана для префиксных кодов}

\paragraph{Коды с минимальной избыточностью}

\paragraph{Самокорректирующиеся коды, границы их мощности}

\paragraph{Коды Хэмминга}

\end{document}

