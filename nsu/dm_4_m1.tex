\documentclass[russian]{article}
\usepackage[T1]{fontenc}
\usepackage[utf8]{inputenc}
\usepackage{amssymb}
\usepackage{esint}
\usepackage{babel}
\usepackage{qtree}
\makeatletter
\makeatother
\pagestyle{myheadings}
\markright{Александров А., 3372}

\begin{document}

1.1. 17,7,27,1,43,3,56,2,29,4,40,32,5,8,28,6

а) алгоритм фон Неймана ($n\log{n}$)

17,7|27,1|43,3|56,2|29,4|40,32|5,8|28,6

7,17,1,27|3,43,2,56|4,29,32,40|5,8,6,28

1,2,3,7,17,27,43,56|4,5,6,8,28,29,32,40

1,2,3,4,5,6,7,8,17,27,28,29,32,40,43,56

б) пирамидальная сортировка ($n\log{n}$)

17,7,27,1,43,3,56,(2),29,4,40,32,5,8,28,(6)

17,7,27,1,43,3,(8),2,29,4,40,32,5,([56],28),6

17,7,27,1,43,(3),8,2,29,4,40,(32,5),56,28,6

17,7,27,1,(4),3,8,2,29,([43],40),32,5,56,28,6

17,7,27,(1),4,3,8,(2,29),43,40,32,5,56,28,6

17,7,(3),1,4,([27],8),2,29,43,40,32,5,56,28,6

17,7,3,1,4,(5),8,2,29,43,40,(32,[27]),56,28,6

17,(1),3,([7],4),5,8,2,29,43,40,32,27,56,28,6

17,1,3,(2),4,5,8,([7],29),43,40,32,27,56,28,6

17,1,3,2,4,5,8,(6),29,43,40,32,27,56,28,([7])

(1),([17],3),2,4,5,8,6,29,43,40,32,27,56,28,7

1,(2),3,([17],4),5,8,6,29,43,40,32,27,56,28,7

1,2,3,(6),4,5,8,([17],29),43,40,32,27,56,28,7

1,2,3,6,4,5,8,(7),29,43,40,32,27,56,28,([17])

1|17,2,3,6,4,5,8,7,29,43,40,32,27,56,28

1|(2),([17],3),6,4,5,8,7,29,43,40,32,27,56,28

1|2,(4),3,(6,[17]),5,8,7,29,43,40,32,27,56,28

1|2,4,3,6,(17),5,8,7,29,(43,40),32,27,56,28

1,2|28,4,3,6,17,5,8,7,29,43,40,32,27,56

1,2|(3),(4,[28]),6,17,5,8,7,29,43,40,32,27,56

1,2|3,4,(5),6,17,([28],8),7,29,43,40,32,27,56

1,2|3,4,5,6,17,(27),8,7,29,43,40,(32,[28]),56

1,2,3|56,4,5,6,17,27,8,7,29,43,40,32,28

1,2,3|(4),([56],5),6,17,27,8,7,29,43,40,32,28

1,2,3|4,(6),5,([56],17),27,8,7,29,43,40,32,28

1,2,3|4,6,5,(7),17,27,8,([56],29),43,40,32,28

1,2,3,4|28,6,5,7,17,27,8,56,29,43,40,32

1,2,3,4|(5),(6,[28]),7,17,27,8,56,29,43,40,32

1,2,3,4|5,6,(8),7,17,(27,[28]),56,29,43,40,32

1,2,3,4,5|32,6,8,7,17,27,28,56,29,43,40

1,2,3,4,5|(6),([32],8),7,17,27,28,56,29,43,40

1,2,3,4,5|6,(7),8,([32],17),27,28,56,29,43,40

1,2,3,4,5|6,7,8,(29),17,27,28,(56,[32]),43,40

1,2,3,4,5,6|40,7,8,29,17,27,28,56,32,43

1,2,3,4,5,6|(7),([40],8),29,17,27,28,56,32,43

1,2,3,4,5,6|7,(17),8,(29,[40]),27,28,56,32,43

1,2,3,4,5,6|7,17,8,29,(40),27,28,56,32,(43)

1,2,3,4,5,6,7|43,17,8,29,40,27,28,56,32

1,2,3,4,5,6,7|(8),(17,[43]),29,40,27,28,56,32

1,2,3,4,5,6,7|8,17,(27),29,40,([43],28),56,32

1,2,3,4,5,6,7,8|32,17,27,29,40,43,28,56

1,2,3,4,5,6,7,8|(17),([32],27),29,40,43,28,56

1,2,3,4,5,6,7,8|17,(29),27,([32],40),43,28,56

1,2,3,4,5,6,7,8|17,29,27,(32),40,43,28,(56)

1,2,3,4,5,6,7,8,17|56,29,27,32,40,43,28

1,2,3,4,5,6,7,8,17|(27),(29,[56]),32,40,43,28

1,2,3,4,5,6,7,8,17|27,29,(28),32,40,(43,[56])

1,2,3,4,5,6,7,8,17,27|56,29,28,32,40,43

1,2,3,4,5,6,7,8,17,27|(28),(29,[56]),32,40,43

1,2,3,4,5,6,7,8,17,27|28,29,(43),32,40,([56])

1,2,3,4,5,6,7,8,17,27,28|56,29,43,32,40

1,2,3,4,5,6,7,8,17,27,28|(29),([56],43),32,40

1,2,3,4,5,6,7,8,17,27,28|29,(32),43,([56],40)

1,2,3,4,5,6,7,8,17,27,28,29|40,32,43,56

1,2,3,4,5,6,7,8,17,27,28,29|(32),([40],43),56

1,2,3,4,5,6,7,8,17,27,28,29|32,(40),43,(56)

1,2,3,4,5,6,7,8,17,27,28,29,32|56,40,43

1,2,3,4,5,6,7,8,17,27,28,29,32|(40),([56],43)

1,2,3,4,5,6,7,8,17,27,28,29,32,40|43,56

1,2,3,4,5,6,7,8,17,27,28,29,32,40|(43),(56)

1,2,3,4,5,6,7,8,17,27,28,29,32,40,43|56

1,2,3,4,5,6,7,8,17,27,28,29,32,40,43,56

\pagebreak

1.2 а)

вставляем 29 : 

\Tree [.29 ]

вставляем 19 : 

\Tree [.29 [.19 ] { } ]

вставляем 9 : 

\Tree [.29 [.19 [.9 ] { } ] { } ]

простой левый поворот (29): 
\Tree [.19 [.9 ] [.29 ] ]

вставляем 1 : 

\Tree [.19 [.9 [.1 ] { } ] [.29 ] ]

вставляем 4 : 

\Tree [.19 [.9 [.1 { } [.4 ] ] { } ] [.29 ] ]

двойной левый поворот (9): 
\Tree [.19 [.4 [.1 ] [.9 ] ] [.29 ] ]

вставляем 14 : 

\Tree [.19 [.4 [.1 ] [.9 { } [.14 ] ] ] [.29 ] ]

двойной левый поворот (19): 
\Tree [.9 [.4 [.1 ] { } ] [.19 [.14 ] [.29 ] ] ]

вставляем 39 : 

\Tree [.9 [.4 [.1 ] { } ] [.19 [.14 ] [.29 { } [.39 ] ] ] ]

вставляем 18 : 

\Tree [.9 [.4 [.1 ] { } ] [.19 [.14 { } [.18 ] ] [.29 { } [.39 ] ] ] ]

вставляем 36 : 

\Tree [.9 [.4 [.1 ] { } ] [.19 [.14 { } [.18 ] ] [.29 { } [.39 [.36 ] { } ] ] ] ]

двойной правый поворот (29): 
\Tree [.9 [.4 [.1 ] { } ] [.19 [.14 { } [.18 ] ] [.36 [.29 ] [.39 ] ] ] ]

вставляем 24 : 

\Tree [.9 [.4 [.1 ] { } ] [.19 [.14 { } [.18 ] ] [.36 [.29 [.24 ] { } ] [.39 ] ] ] ]

простой правый поворот (9): 
\Tree [.19 [.9 [.4 [.1 ] { } ] [.14 { } [.18 ] ] ] [.36 [.29 [.24 ] { } ] [.39 ] ] ]

вставляем 15 : 

\Tree [.19 [.9 [.4 [.1 ] { } ] [.14 { } [.18 [.15 ] { } ] ] ] [.36 [.29 [.24 ] { } ] [.39 ] ] ]

двойной правый поворот (14): 
\Tree [.19 [.9 [.4 [.1 ] { } ] [.15 [.14 ] [.18 ] ] ] [.36 [.29 [.24 ] { } ] [.39 ] ] ]

вставляем 12 : 

\Tree [.19 [.9 [.4 [.1 ] { } ] [.15 [.14 [.12 ] { } ] [.18 ] ] ] [.36 [.29 [.24 ] { } ] [.39 ] ] ]

б) 

простой правый поворот ($b(A) = 2, b(B) \ne -1$):

\Tree [.A L [.B C R ] ]
\Tree [.B [.A L C ] R ]

двойной правый поворот ($b(A) = 2, b(B) = -1$):

\Tree [.A L [.B [.C M N ] R ] ]
\Tree [.C [.A L M ] [.B N R ] ]

простой левый поворот ($b(A) = -2, b(B) \ne 1$):

\Tree [.A [.B L C ] R ]
\Tree [.B L [.A C R ] ]

двойной левый поворот ($b(A) = -2, b(B) = 1$):

\Tree [.A [.B L [.C M N ] ] R ]
\Tree [.C [.B L M ] [.A N R ] ]

удаление: если вершина лист, удаляем её и делаем балансировку всех родителей до корня. если нет, найдём самую близкую по значению вершину в поддереве наибольшей высоты и удалим её, поставив на место удаляемой.

удаляем 24 : 

\Tree [.19 [.9 [.4 [.1 ] { } ] [.15 [.14 [.12 ] { } ] [.18 ] ] ] [.36 [.29 ] [.39 ] ] ]

двойной левый поворот (19): 
\Tree [.15 [.9 [.4 [.1 ] { } ] [.14 [.12 ] { } ] ] [.19 [.18 ] [.36 [.29 ] [.39 ] ] ] ]

удаляем 12 : 

\Tree [.15 [.9 [.4 [.1 ] { } ] [.14 ] ] [.19 [.18 ] [.36 [.29 ] [.39 ] ] ] ]

\pagebreak

1.4 $P = \prod_i (1-(1-\frac{i}{i+1})^{n_i}) = \prod_i(1-\frac{1}{(i+1)^{n_i}}) $ при $\sum n_i \le 8$

$P_k(y)=max_{0 \le t \le y} P_{k-1}(y-t)(1-\frac{1}{(k+1)^t})$

\begin{tabular}{|c|c|c|c|c|c|c|c|c|c|}\hline
 & 0
 & 1
 & 2
 & 3
 & 4
 & 5
 & 6
 & 7
 & 8
\\\hline
$P_{1}$
 & 0 / 0
 & $\frac{1}{2}$ / 1
 & $\frac{3}{4}$ / 2
 & $\frac{7}{8}$ / 3
 & $\frac{15}{16}$ / 4
 & $\frac{31}{32}$ / 5
 & $\frac{63}{64}$ / 6
 & $\frac{127}{128}$ / 7
 & $\frac{255}{256}$ / 8
\\\hline
$P_{2}$
 & 0 / 0
 & 0 / 0
 & $\frac{1}{3}$ / 1
 & $\frac{1}{2}$ / 1
 & $\frac{2}{3}$ / 2
 & $\frac{7}{9}$ / 2
 & $\frac{91}{108}$ / 3
 & $\frac{65}{72}$ / 3
 & $\frac{403}{432}$ / 3
\\\hline
$P_{3}$
 & 0 / 0
 & 0 / 0
 & 0 / 0
 & $\frac{1}{4}$ / 1
 & $\frac{3}{8}$ / 1
 & $\frac{1}{2}$ / 1
 & $\frac{5}{8}$ / 2
 & $\frac{35}{48}$ / 2
 & $\frac{455}{576}$ / 2
\\\hline
$P_{4}$
 & 0 / 0
 & 0 / 0
 & 0 / 0
 & 0 / 0
 & $\frac{1}{5}$ / 1
 & $\frac{3}{10}$ / 1
 & $\frac{2}{5}$ / 1
 & $\frac{1}{2}$ / 1
 & $\frac{3}{5}$ / 2
\\\hline
\end{tabular}

$P=\frac{3}{5}$, $n=(2,2,2,2)$

1.5 $max\Sigma f_i (x_i)$ при $\Sigma x_i \le 9$

$S_k(y)=max_{0 \ge t \ge y} ( S_{k-1}(y-t) + f_k(t))$

найти $S_4(9)$

\begin{tabular}{|c|c|c|c|c|c|c|c|c|c|c|}\hline
		& 0 	& 1 	& 2 	& 3 	& 4		& 5 	& 6 	& 7 	& 8 	& 9  	\\\hline
$S_1$	& 2 / 0 & 2 / 1 & 3 / 2 & {\bf 5 / 3}	& 5 / 3 & 5 / 3 & 5 / 3 & 5 / 3 & 5 / 3 & 5 / 3 \\\hline
$S_2$	& 4 / 0 & 6 / 1 & 6 / 1 & 7 / 1 & {\bf 9 / 1 }& 9 / 1 & 9 / 1 & 9 / 1 & 9 / 1 & 9 / 1 \\\hline
$S_3$	& 6 / 0 & 8 / 1 & 9 / 0 & 11 / 0 & 11 / 0 & 12 / 2 & 14 / 3 & {\bf 14 / 3} & 14 / 3 & 14 / 3 \\\hline
$S_4$	& 8 / 0 & 10 / 0 & 11 / 1 & 12 / 1 & 13 / 0 & 14 / 1 & 15 / 1 & 16 / 0 & 17 / 1 & {\bf 18 / 2} \\\hline
\end{tabular}

$S = 18$, $x = (3, 1, 3, 2)$

1.6 решаем обратную задачу: минимизируем количество сырья для заданной прибыли

$Q_k(y)$ - минимальное количество сырья для получения прибыли $y$ от первых $k$ видов продукции

$Q_k(y)=min_{0 \le t \le [\frac{y}{c_k}]}(Q_{k-1}(y-t c_k) + S_k(t))$

{
\fontsize{6}{8}
\selectfont
\begin{tabular}{|@{}c@{}|@{}c@{}|@{}c@{}|@{}c@{}|@{}c@{}|@{}c@{}|@{}c@{}|@{}c@{}|@{}c@{}|@{}c@{}|@{}c@{}|@{}c@{}|@{}c@{}|@{}c@{}|@{}c@{}|@{}c@{}|@{}c@{}|@{}c@{}|@{}c@{}|@{}c@{}|@{}c@{}|@{}c@{}|@{}c@{}|@{}c@{}|@{}c@{}|@{}c@{}|}\hline
      & 0 & 1      & 2      & 3      & 4      & 5      & 6      & 7      & 8      & 9      & 10     & 11     & 12     & 13     & 14      & 15      & 16 & 17 & 18 & 19 & 20 & 21 & 22 & 23 & 24 \\\hline
$Q_1$ & 0/0 & 11/1 & 11/1 & {\bf 11/1} & 23/2 & 23/2 & 23/2 & 37/3  & 37/3  & 37/3 & 44/4 & 44/4 & 44/4 & 47/5 & 47/5 & 47/5 &   &     &&&&&&&  \\\hline
$Q_2$ & 0/0 & 11/0 & 11/0 & 11/0 & 21/1 & 23/0 & 23/0 & 29/2 & 29/2 & 37/0 & 38/3 & 38/3 & 38/3 & 40/4 & 40/4 & 40/4 & 40/4 & 51/4 & 51/4 & {\bf 51/4} & 61/5 & 63/4 & 63/4 & 72/5 &\\\hline
$Q_2$ & & & & & & & & & & & & & & & & & & & & & & & 63/0 & {\bf 65/2} & 68/4 \\\hline\end{tabular}
}

$n=(1,4,2)$

прибыль: 23

расходы: 65

1.7 решаем обратную задачу: минимизируем вес для получения заданной стоимости

$Q_k(y)$ - минимум веса для набора стоимости $y$ из первых $k$ предметов

$Q_k(y) = min\{Q_{k-1}(y); Q_{k-1}(y-c_k)+p_k\}$

\begin{tabular}{|@{}c@{}|@{}c@{}|@{}c@{}|@{}c@{}|@{}c@{}|@{}c@{}|@{}c@{}|@{}c@{}|@{}c@{}|@{}c@{}|@{}c@{}|@{}c@{}|@{}c@{}|@{}c@{}|@{}c@{}|@{}c@{}|@{}c@{}|@{}c@{}|@{}c@{}|}\hline
      & 0 & 1      & 2      & 3      & 4      & 5      & 6      & 7      & 8      & 9      & 10     & 11     & 12     & 13     & 14      & 15      & 16 & 17 \\\hline
$Q_1$ & {\bf 0/0} & 10/1 &        &        &        &        &        &        &        &        &        &        &        &        &         &         &         & \\\hline
$Q_2$ & 0/0 & 10/0 & {\bf 13/1} & 23/1 &        &        &        &        &        &        &        &        &        &        &         &         &         & \\\hline
$Q_3$ & 0/0 & 10/0 & 13/0 & 21/1 & 33/1 & {\bf 34/1} & 44/1 &        &        &        &        &        &        &        &         &         &         & \\\hline
$Q_4$ & 0/0 & 10/0 & 13/0 & 21/0 & 28/1 & 34/0 & 41/1 & 49/1 & 59/1 & {\bf 62/1 } & 72/1 &        &        &        &         &         &         & \\\hline
$Q_5$ & 0/0 & 10/0 & 13/0 & 21/0 & 28/0 & 34/0 & 41/0 & 49/0 & 59/0 & {\bf 62/0 } & 72/0 & 81/1 & 89/1 & 99/1 & 102/1 & 112/1 &         & \\\hline
$Q_6$ & 0/0 & 10/0 & 13/0 & 21/0 & 28/0 & 34/0 & 41/0 & 49/0 & 59/0 & {\bf 62/0 }& 72/0 & 81/0 & 89/0 & 98/1 & 102/0 & 111/1 & 121/1 & \\\hline
$Q_7$ & 0/0 & 10/0 & 13/0 & 21/0 & 28/0 & 34/0 & 41/0 & 49/0 & 59/0 & 62/0 & 72/0 & 81/0 & 89/0 & 97/1 & 102/0 & 111/0 & {\bf 118/1} & 128/0 \\\hline
\end{tabular}

берем: 2,3,4,7

стоимость: 16

вес: 118

\end{document} 