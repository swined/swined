\documentclass[russian]{article}
\usepackage[T1]{fontenc}
\usepackage[utf8]{inputenc}
\usepackage{amsmath,amssymb,esint,babel,qtree,ulem}
\usepackage{graphicx}
\makeatletter
\makeatother
\pagestyle{myheadings}
\markright{Александров А., 3372}

\begin{document}

\section*{3.1}

$z=x_1 + 2 x_2 \to \max$, $x_i \ge 0$

$\begin{cases}
-2 x_1 + x_2 + x_3 \ge 1 \\
x_1 - 2 x_2 + 3 x_4 = -2 \\
5 x_1 + x_2 + x_5 \le 3
\end{cases}$

приводим к каноническому виду:

$w = -z = -x _1 - 2x_2 \to \min$

$\begin{cases}
-2 x_1 + x_2 + x_3 - 1 = x_6 \\
x _1 - 2 x_3 + 3 x_4 = -2 \\
-5 x_1 - x_2 - x_5 + 3 = x_7 
\end{cases}$

$\begin{cases}
-2 x_1 + x_2 + x_3 - x_6 = 1 \\
x _1 - 2 x_3 + 3 x_4 = -2 \\
5 x_1 + x_2 + x_5 + x_7 = 3
\end{cases}$

выражаем через $\{x_1, x_5, x_7\}$

$x_2 = 3 - 5 x_1 - x_5 - x_7$

$w = 9 x_1 + x_5 + x_7 - 10$

приводим к базису $\{x_2, x_4, x_6\}$

\begin{tabular}{|c|ccccccc|}\hline
& $x_1$ & $x_2$ & $x_3$ & $x_4$ & $x_5$ & $x_6$ & $x_7$ \\\hline
1 & -2 & 1 & 1 & 0 & 0 & -1 & 0 \\
-1 & 1 & -2 & 0 & 3 & 0 & 0 & 0 \\
3 & 5 & 1 & 0 & 0 & 1 & 0 & 1 \\\hline

0 & 3 & 0 & 2 & 0 & 0 & -2 & 0 \\
-1 & 11 & 0 & 0 & 3 & 2 & 0 & 2 \\
3 & 5 & 1 & 0 & 0 & 1 & 0 & 1 \\\hline

3 & 5 & 1 & 0 & 0 & 1 & 0 & 1 \\
$-\frac{1}{3}$ & $\frac{11}{3}$ & 0 & 0 & 1 & $\frac{2}{3}$ & 0 & $\frac{2}{3}$ \\
0 & $-\frac{3}{2}$ & 0 & -1 & 0 & 0 & 1 & 0 \\\hline
\end{tabular}

положим $x_1=x_5 = x_7 =0$, тогда $x_2 = 3, x_4 = - \frac{1}{3}, x_6 - x_3 = 0 \Rightarrow \min w = -10 \Rightarrow \min z = 10$

\paragraph{б}

$z = x_2 - x_1 \to \min$

$\begin{cases}
-2 x_1 + x_2 + x_3 = 2 \\
x_1 - 2x_2 + 2 x_4 = 4 \\
x_1 + x_2 + 2 x_5 = 5
\end{cases}$

выражаем через $\{x_4, x_5\}$

$ x_2 = \frac{1}{3}(1  + 2 x_4 - 2 x_5) , x_1 = \frac{1}{3}(14 - 2 x+4 - 4 x+5)$

$z = \frac{1}{3}(4 x_4 + 2 x_5 - 13)$

переходим к базису $\{ x_1, x_2, x_3\}$

\begin{tabular}{|c|ccccc|}\hline
& $x_1$ & $x_2$ & $x_3$ & $x_4$ & $x_5$ \\\hline

2 & -2 & 1 & 3 & 0 & 0 \\
4 & 1 & -2 & 0 & 2 & 0 \\
5 & 1 & 1 & 0 & 0 & 2 \\\hline

11 & 0 & 0 & 3 & 2 & 2 \\
-1 & 0 & -3 & 0 & 2 & -2 \\
14 & 2 & 0 & 0 & 2 & 4  \\\hline

7 & 1 & 0 & 0 & 1 & 2 \\
$\frac{1}{3}$ & 0 & 1 & 0 & $-\frac{2}{3}$ & $\frac{2}{3}$ \\
$\frac{14}{3}$ & 0 & 0 & 1 & $\frac{2}{3}$ & $\frac{2}{3}$ \\\hline

\end{tabular}

положим $x_4=x_5 = 0 \Rightarrow x_1 = 7, x_2 = \frac{1}{3},x_3 = \frac{11}{3} \Rightarrow \min z = -13$

\section*{3.4}

$\{aa,ab,cc,cca,bcca\}$

\paragraph{а}

\Tree [.$ccabccabccabcc$ [.$cc$ [.$ab$ [.$cc$ [.$ab$ [.$cc$ [.$ab$ [.$cc$  ] ] ] [.$cca$ \sout{$bcc$} ] ] ] [.$cca$ [.$bcca$ \sout{$bcc$} ] ] ] ] [.$cca$ [.$bcca$ [.$bcca$ \sout{$bcc$} ] ] ] ]

\paragraph{б}

\Tree [.$bccaccabccabccacabcca$ [.$bcca$ [.$cc$ [.$ab$ [.$cc$ [.$ab$ [.$cc$ \sout{$acabcca$} ] [.$cca$ \sout{$cabcca$} ] ] ] [.$cca$ [.$bcca$ \sout{$cabcca$} ] ] ] ] [.$cca$ [.$bcca$ [.$bcca$ \sout{$cabcca$} ] ] ] ] ]

\paragraph{в}

\Tree [.$abbccaccabccaabab$ [.$ab$ [.$bcca$ [.$cc$ [.$ab$ [.$cc$ [.$aa$ \sout{$bab$} ] ] [.$cca$ [.$ab$ [.$ab$  ] ] ] ] ] [.$cca$ [.$bcca$ [.$ab$ [.$ab$  ] ] ] ] ] ] ]

\section*{3.9}

пусть $C_1 \cup C_2 / \{e\}$ - зависимо. тогда в множестве $C_1 \cup C_2$ есть независимое $B$ такое, что $|B|=|C_1 \cup C_2| - 1$. если $A=C_1 \cap C_2$, то $A$ - часть цикла, следовательно $A$ - независимое, $|A| < |B|$. по 2 аксиоме можно дополнить его, т.к. максимальные по включению множества имеют одну размерность. тогда получается независимое множество, которое содержит либо цикл $C_1$, либо цикл $C_2$, а значит оно зависимое - противоречие.

\end{document} 