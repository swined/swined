\documentclass[russian]{article}
\usepackage[T1]{fontenc}
\usepackage[utf8]{inputenc}
\usepackage{amsmath,amssymb,esint,babel,qtree,ulem}
\usepackage{graphicx}
\makeatletter
\makeatother
\pagestyle{myheadings}
\markright{Александров А., 3372}

\begin{document}

\section*{3.1}

$z=x_1 + 2 x_2 \to \max$, $x_i \ge 0$

$\begin{cases}
-2 x_1 + x_2 + x_3 \ge 1 \\
x_1 - 2 x_2 + 3 x_4 = -2 \\
5 x_1 + x_2 + x_5 \le 3
\end{cases}$

приводим к каноническому виду:

$w = -z = -x _1 - 2x_2 \to \min$

$\begin{cases}
-2 x_1 + x_2 + x_3 - 1 = x_6 \\
x _1 - 2 x_3 + 3 x_4 = -2 \\
-5 x_1 - x_2 - x_5 + 3 = x_7 
\end{cases}$

$\begin{cases}
-2 x_1 + x_2 + x_3 - x_6 = 1 \\
x _1 - 2 x_3 + 3 x_4 = -2 \\
5 x_1 + x_2 + x_5 + x_7 = 3
\end{cases}$

выражаем через $\{x_1, x_5, x_7\}$

$x_2 = 3 - 5 x_1 - x_5 - x_7$

$w = 9 x_1 + x_5 + x_7 - 10$

приводим к базису $\{x_2, x_4, x_6\}$

\begin{tabular}{|c|ccccccc|}\hline
& $x_1$ & $x_2$ & $x_3$ & $x_4$ & $x_5$ & $x_6$ & $x_7$ \\\hline
1 & -2 & 1 & 1 & 0 & 0 & -1 & 0 \\
-1 & 1 & -2 & 0 & 3 & 0 & 0 & 0 \\
3 & 5 & 1 & 0 & 0 & 1 & 0 & 1 \\\hline

0 & 3 & 0 & 2 & 0 & 0 & -2 & 0 \\
-1 & 11 & 0 & 0 & 3 & 2 & 0 & 2 \\
3 & 5 & 1 & 0 & 0 & 1 & 0 & 1 \\\hline

3 & 5 & 1 & 0 & 0 & 1 & 0 & 1 \\
$-\frac{1}{3}$ & $\frac{11}{3}$ & 0 & 0 & 1 & $\frac{2}{3}$ & 0 & $\frac{2}{3}$ \\
0 & $-\frac{3}{2}$ & 0 & -1 & 0 & 0 & 1 & 0 \\\hline
\end{tabular}

положим $x_1=x_5 = x_7 =0$, тогда $x_2 = 3, x_4 = - \frac{1}{3}, x_6 - x_3 = 0 \Rightarrow \min w = -10 \Rightarrow \min z = 10$

\paragraph{б}

$z = x_2 - x_1 \to \min$

$\begin{cases}
-2 x_1 + x_2 + x_3 = 2 \\
x_1 - 2x_2 + 2 x_4 = 4 \\
x_1 + x_2 + 2 x_5 = 5
\end{cases}$

выражаем через $\{x_4, x_5\}$

$ x_2 = \frac{1}{3}(1  + 2 x_4 - 2 x_5) , x_1 = \frac{1}{3}(14 - 2 x+4 - 4 x+5)$

$z = \frac{1}{3}(4 x_4 + 2 x_5 - 13)$

переходим к базису $\{ x_1, x_2, x_3\}$

\begin{tabular}{|c|ccccc|}\hline
& $x_1$ & $x_2$ & $x_3$ & $x_4$ & $x_5$ \\\hline

2 & -2 & 1 & 3 & 0 & 0 \\
4 & 1 & -2 & 0 & 2 & 0 \\
5 & 1 & 1 & 0 & 0 & 2 \\\hline

11 & 0 & 0 & 3 & 2 & 2 \\
-1 & 0 & -3 & 0 & 2 & -2 \\
14 & 2 & 0 & 0 & 2 & 4  \\\hline

7 & 1 & 0 & 0 & 1 & 2 \\
$\frac{1}{3}$ & 0 & 1 & 0 & $-\frac{2}{3}$ & $\frac{2}{3}$ \\
$\frac{14}{3}$ & 0 & 0 & 1 & $\frac{2}{3}$ & $\frac{2}{3}$ \\\hline

\end{tabular}

положим $x_4=x_5 = 0 \Rightarrow x_1 = 7, x_2 = \frac{1}{3},x_3 = \frac{11}{3} \Rightarrow \min z = -13$

\section*{3.2}

\paragraph{а}

$\begin{cases}
-1 = 2 x_1 - x_2 + x_3 - 2 x_4 + x_5 + x_6 \\
-2 = -3 x_1 + x_2 + x_4 - x_5 + x_6 \\
-3 = - 5 x_1 + x_2 - 2 x_3 + x_4 - x_6
\end{cases}$

$u_1 + u_2 + u_3 \to \min$

$\begin{cases}
1 =u_1 - 2 x_1 + x_2 - x_3 + 2 x_4 - x_5 - x_6\\
2 =u_2 + 3 x_1 - x_2 - x_4 + x_5 - x_6\\
3 =u_3 +  5 x_1 - x_2 + 2 x_3 - x_4 + x_6
\end{cases}$

$\begin{cases}
1 = - 2 x_1 + x_2 - x_3 + 2 x_4 - x_5 - x_6 + u_1\\
2 = 3 x_1 - x_2 - x_4 + x_5 - x_6 + u_2\\
3 = 5 x_1 - x_2 + 2 x_3 - x_4 + x_6 + u_3
\end{cases}$

$f=6 - 6 x_1 + x_2 - x_3 + x_6$

$-6=-f - 6 x_1 + x_2 - x_3 + x_6$

\begin{tabular}{|c|c|ccccccccc|} \hline
 & & $u_1$ & $u_2$ & $u_3$ & $x_1$ & $x_2$ & $x_3$ & $x_4$ & $x_5$ & $x_6$  \\\hline

0 & -6 & 0 & 0 & 0 & $(-6)$ & 1 & -1 & 0 & 0 & 1 \\
$u_1$ & 1 & 1 & 0 & 0 & -2 & 1 & -1 & 2 & -1 & -1 \\
$u_2$ & 2 & 0 & 1 & 0 & 3 & -1 & 0 & -1 & 1 & -1 \\
$(u_3)$ & 3 & 0 & 0 & 1 & $(5)$ & -1 & 2 & -1 & 0 & 1 \\\hline

0 & $-\frac{12}{5}$ & 0 & 0 & $\frac{6}{5}$ & 0 & $-\frac{1}{5}$ & $\frac{7}{5}$ & $(-\frac{6}{5})$ & 0 &  $\frac{11}{5}$ \\
$(u_1)$ & $\frac{11}{5}$ & 1 & 0 & $\frac{2}{5}$ & 0 & $\frac{3}{5}$ & $-\frac{1}{5}$ & $(\frac{8}{5})$ & -1 & $-\frac{3}{5}$ \\
$u_2$ & $-\frac{1}{5}$ & 0 & 1 & $-\frac{3}{5}$ & 0 & $-\frac{2}{5}$ & $-\frac{6}{5}$ & $-\frac{2}{5}$ & 1 & $-\frac{8}{5}$ \\
$x_1$ & $\frac{3}{5}$ & 0 & 0 & $\frac{1}{5}$ & 1 & $-\frac{1}{5}$ & $\frac{2}{5}$ & $-\frac{1}{5}$ & 0 & $\frac{1}{5}$ \\\hline

0 & $-\frac{3}{4}$ & $\frac{3}{4}$ & 0 & $\frac{9}{10}$ & 0 & $\frac{1}{4}$ & $\frac{5}{4}$ & 0 & $(-\frac{3}{4})$ & $\frac{7}{4}$ \\
$x_4$ & $\frac{11}{8}$ & $\frac{5}{8}$ & 0 & $\frac{1}{4}$ & 0 & $\frac{3}{8}$ & $-\frac{1}{8}$ & 1 & $-\frac{5}{8}$ & $-\frac{3}{8}$ \\
$(u_2)$ & & & & & & & & & $(\frac{3}{4})$ & \\
$x_1$ & & & & & & & & & $-\frac{1}{8}$ & \\\hline
\end{tabular}

допустимый базис - $\{x_1, x_4, x_5\}$

\paragraph{б}

$f=0,\{u_1, u_2, x_3\}$

\begin{tabular}{|c|c|cccccccc|} \hline
 & & $x_1$ & $x_2$ & $x_3$ & $x_4$ & $x_5$ & $u_1$ & $u_2$ & $u_3$ \\\hline
$u_1$ & 0 & 1 & 0 & 0 & 3 & 0 & 1 & 0 & 0 \\
$u_2$ & 0 & 0 & 0 & 0 & 1 & 3 & 0 & 1 & 0 \\
$x_3$ & 7 & 5 & -2 & 1 & 1 & 0 & 0 & 0 & 1 \\\hline
\end{tabular}

$f=u_1 + u_2 + x_3$

$f=7 - u_3 - 6 x_1 + 2 x_2 - 3 x_4 - 3 x_5 $

\begin{tabular}{|c|c|cccccccc|} \hline
 & & $x_1$ & $x_2$ & $x_3$ & $x_4$ & $x_5$ & $u_1$ & $u_2$ & $u_3$ \\\hline
$u_1$ & 0 & 1 & 0 & 0 & 3 & 0 & 1 & 0 & 0 \\
$u_2$ & 0 & 0 & 0 & 0 & 1 & 3 & 0 & 1 & 0 \\
$u_3$ & 7 & 0 & -2 & 1 & -14 & 0 & -5 & 0 & 1 \\\hline
\end{tabular}

$x_1 = -3 x_4 - u_1 $

$f = 7 - u_3 + 18 x_4 + 6 u_1 + 2 x_2 - 5 x_2 - 5 x_4 - 3 x_5$

$f = 7 - u_3 + 13 x_4 + 2 x_2 - 3 x_5 + 6 u_1$

$\{x_1, u_2, x_3\}$

\begin{tabular}{|c|c|cccccccc|} \hline
 & & $x_1$ & $x_2$ & $x_3$ & $x_4$ & $x_5$ & $u_1$ & $u_2$ & $u_3$ \\\hline
$u_1$ & 0 & 1 & 0 & 0 & 3 & 0 & 1 & 0 & 0 \\
$u_2$ & 0 & 0 & 0 & 0 & 1 & 3 & 0 & 1 & 0 \\
$u_3$ & 7 & 0 & -2 & 1 & 0 & 42 & 0 & 14 & 1 \\\hline
\end{tabular}

$x_4 = -3 x_5 - u_2$

$f=7-u_3 + 13 x_4 + 2 x_2 - 3 x_5 + 6 u_1 = x_3 + u_2 + 6 u_1$

\section*{3.4}

$\{aa,ab,cc,cca,bcca\}$

\paragraph{а}

\Tree [.$ccabccabccabcc$ [.$cc$ [.$ab$ [.$cc$ [.$ab$ [.$cc$ [.$ab$ [.$cc$  ] ] ] [.$cca$ \sout{$bcc$} ] ] ] [.$cca$ [.$bcca$ \sout{$bcc$} ] ] ] ] [.$cca$ [.$bcca$ [.$bcca$ \sout{$bcc$} ] ] ] ]

\paragraph{б}

\Tree [.$bccaccabccabccacabcca$ [.$bcca$ [.$cc$ [.$ab$ [.$cc$ [.$ab$ [.$cc$ \sout{$acabcca$} ] [.$cca$ \sout{$cabcca$} ] ] ] [.$cca$ [.$bcca$ \sout{$cabcca$} ] ] ] ] [.$cca$ [.$bcca$ [.$bcca$ \sout{$cabcca$} ] ] ] ] ]

\paragraph{в}

\Tree [.$abbccaccabccaabab$ [.$ab$ [.$bcca$ [.$cc$ [.$ab$ [.$cc$ [.$aa$ \sout{$bab$} ] ] [.$cca$ [.$ab$ [.$ab$  ] ] ] ] ] [.$cca$ [.$bcca$ [.$ab$ [.$ab$  ] ] ] ] ] ] ]

\section*{3.6}

\paragraph{а}

рассмотрим 10-мерную булеву решетку. разделим все ее вершины на 2 группы: четные и нечетные слои. утверждаем, что в этих группах любые 2 элемента ($e_1,e_2$) из 1 группы имеют $\ge 2$ различающихся координат.

1) $e_1$ и $e_2$ принадлежат одному слою. очевидно, у них будет $\ge 2$ различающихся координат. например $(0,0,...,0,1,1)$ и $(0,0,...,1,0,1)$

2) $e_1$ и $e_2$ принадлежат разным слоям. т.к. в группу входят слои через один, то $e_1$ и $e_2$ также имеют не менне двух различающихся координат. например $(0,...,0,0)$ и $(0,...,1,1)$.

\paragraph{б} 

рассмотрим векторы длины $n$ с люыми тремя различными позициями. или, что то же самое - все возможные коды Хэмминга с кодовым расстоянием 3. мощность такого множества $f(n,d)\le\frac{2^n}{|V_{\lfloor \frac{d-1}{2} \rfloor}|} = \frac{2^n}{1+C_n^1}=\frac{2^n}{1+n}$. т.е. нельзя выбрать более $\frac{2^n}{n+1}$ $n$-значных чисел так, что в сумме любых двух из них содержится не менее  3 троек.

\section*{3.7}

число от 1 до 1000 может быть закодировано при помощи 10 бит, в 1 из которых может быть допущена ошибка

\paragraph{а}
последовательно делим интервал пополам. задаем каждый вопрос 2 раза, в случае несовпадения переспрашиваем.

\paragraph{б}

вопросы 1-10: последовательно делим интервал пополам

вопрос 11: повторяем вопросы 1,2,4,5,7,9 и спрашиваем, было ли на них нечетное число ответов да, ставим ответ в позицию (1)

вопрос 12: повторяем вопросы 1,3,4,6,7,9 и спрашиваем, было ли на них нечетное число ответов да, ставим ответ в позицию (2)

вопрос 13: повторяем вопросы 2,3,4,8,9,10 и спрашиваем, было ли на них нечетное число ответов да, ставим ответ в позицию (3)

вопрос 14: повторяем вопросы 5,6,7,8,9,10 и спрашиваем, было ли на них нечетное число ответов да, ставим ответ в позицию (4)

{
\tiny
\begin{tabular}{|c|c|c|c|c|c|c|c|c|c|c|c|c|c|c|} \hline
& (1) & (2) & 1 & (3) & 2 & 3 & 4 & (4) & 5 & 6 & 7 & 8 & 9 & 10 \\\hline
& 0001 & 0010 & 0011 & 0100 & 0101 & 0110 & 0111 & 1000 & 0001 & 1010 & 1011 & 1100 & 1101 & 1110 \\\hline
(1) & & & + & & + &   & + & & + &   & + &   & + &   \\\hline
(2) & & & + & &   & + & + & &   & + & + &   &   & + \\\hline
(3) & & &   & & + & + & + & &   &   &   & + & + & + \\\hline
(4) & & &   & &   &   &   & & + & + & + & + & + & + \\\hline
\end{tabular}
}

получаем код Хэмминга, по которому можно восстановить число

\section*{3.9}

пусть $C_1 \cup C_2 / \{e\}$ - зависимо. тогда в множестве $C_1 \cup C_2$ есть независимое $B$ такое, что $|B|=|C_1 \cup C_2| - 1$. если $A=C_1 \cap C_2$, то $A$ - часть цикла, следовательно $A$ - независимое, $|A| < |B|$. по 2 аксиоме можно дополнить его, т.к. максимальные по включению множества имеют одну размерность. тогда получается независимое множество, которое содержит либо цикл $C_1$, либо цикл $C_2$, а значит оно зависимое - противоречие.

\section*{3.10}

\paragraph{а}

чтобы доказать, что любой матроид разбиения является трансверсальным матроидом надо проверить однозначность соответствия между независимыми множествами в этих матроидах. 

разбиений: множество дуг орграфа независ. $\iff \nexists \delta : \to \delta \leftarrow$

трансверсалей: $G'(U,W,E)$, $U'$ независимо $\iff \exists $ п.с. покр. $U'$

возьмем произв. орграф и построим для него 2-дольный граф $G(U,W,E)$:

$U$ - мн-во вершин

$W$ - мн-во дуг

$E:(u,w) \in E \iff $ ребро $w$ направлено в $u$

в таком случае из каждой верш. в $w$ будет очевидно выходить 1 ребро. при этом независимым мн-во дуг в орграфе будет $\iff$ любые 2 из них не будут входить в 1 вершину. но тогда на верш. в $w$, соотв. этим ребрам будет $\exists$ п.с., т.к. в верш. из $u$ из них будет входить $\le 1$ ребра. и наоборот - $\forall$ п.с. $\to$ независимые дуги.

\paragraph{б}

построим соотв. м-ду независимыми мн-вами в граф. матроиде и в матричном.

построим матрицу инцидентности вершин и ребер графа: 1 стоит на $(i,j)$ месте, если верш. $j \in $ ребру $i$, иначе 0. очевидно, в каждой строке будет всего по 2 единицы, т.к. у ребра 2 конца. в таком случае зависимост мн-ва вершин в графе (т.е. $\exists$ цикла) будет означать, что $\exists$ конструкция в матрице, такая, что когда идем по ребрам, для каждой 1 в строке будет $\exists$ пара на другом конце, т.к. вершина принадл. 2 ребрам $\Rightarrow$ когда сложим все строки над полем 2 чисел $(1+1=0)$, то  кроме нулейдля $\forall 1$ будет пара $\Rightarrow \sum$ получится нулевой строкой, а значит вектора линейно зависимы.

и наоброт, если нет циклов, то не будет набора строк таких, что $\forall 1 \exists ! $ пара $\Rightarrow $ лин. независ.

\end{document} 